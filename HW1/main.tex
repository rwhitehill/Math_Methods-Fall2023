\def\duedate{\today}
\def\HWnum{1}
\documentclass[10pt,a4paper]{book}

% custom section formatting
\usepackage{titlesec}
\titleformat{\chapter}[display]
{\normalfont\Large\filcenter\sffamily}
{\titlerule[1pt]%
\vspace{1pt}%
\titlerule
\vspace{1pc}%
\LARGE\MakeUppercase{\chaptertitlename} \thechapter}
{1pc}
{\titlerule
\vspace{1pc}%
\Huge}

% appendix handling
\usepackage[toc,page]{appendix}
    
% encoding for file and font
\usepackage[utf8]{inputenc}
\usepackage[T1]{fontenc}

% math formatting/tools
\usepackage{amsmath}
\usepackage{amssymb}
\usepackage{mathtools}
\usepackage[arrowdel]{physics}
\usepackage{dsfont}

\newcommand{\R}{\mathbb{R}}
\newcommand{\Z}{\mathbb{Z}}
\newcommand{\N}{\mathbb{N}}
\newcommand{\Q}{\mathbb{Q}}
\newcommand{\C}{\mathbb{C}}

% unit formatting
\usepackage{siunitx}
\AtBeginDocument{\RenewCommandCopy\qty\SI}

% figure formatting/tools
\usepackage{graphicx}
\usepackage{float}
\usepackage{subcaption}
\usepackage{multirow}
\usepackage{import}
\usepackage{pdfpages}
\usepackage{transparent}
\usepackage{currfile}

\NewDocumentCommand\incfig{O{1} m}{
    \def\svgwidth{#1\textwidth}
    \import{./Figures/\currfiledir}{#2.pdf_tex}
}

\newcommand{\bef}{\begin{figure}[h!tb]\centering}
\newcommand{\eef}{\end{figure}}

\newcommand{\bet}{\begin{table}[h!tb]\centering}
\newcommand{\eet}{\end{table}}

% hyperlink references 
\usepackage{hyperref}
\hypersetup{
    colorlinks=true,
    linkcolor=blue,
    filecolor=magenta,
    urlcolor=cyan,
    pdftitle={Physics 1 Notes},
    pdfauthor={Richard Whitehill},
    pdfpagemode=FullScreen
}
\urlstyle{same}

\newcommand{\eref}[1]{Eq.~(\ref{eq:#1})}
\newcommand{\erefs}[2]{Eqs.~(\ref{eq:#1})--(\ref{eq:#2})}

\newcommand{\fref}[1]{Fig.~(\ref{fig:#1})}
\newcommand{\frefs}[2]{Fig.~(\ref{fig:#1})--(\ref{fig:#2})}

\newcommand{\aref}[1]{Appendix~(\ref{app:#1})}
\newcommand{\sref}[1]{Section~(\ref{sec:#1})}
\newcommand{\srefs}[2]{Sections~(\ref{sec:#1})-(\ref{sec:#2})}

\newcommand{\tref}[1]{Table~(\ref{tab:#1})}
\newcommand{\trefs}[2]{Table~(\ref{tab:#1})--(\ref{tab:#2})}

% tcolorbox formatting/definitions
\usepackage[most]{tcolorbox}
\usepackage{xcolor}
\usepackage{xifthen}
\usepackage{parskip}

\definecolor{peach}{rgb}{1.0,0.8,0.64}

\DeclareTColorBox[auto counter, number within=chapter]{defbox}{O{}}{
    enhanced,
    boxrule=0pt,
    frame hidden,
    borderline west={4pt}{0pt}{green!50!black},
    colback=green!5,
    before upper=\textbf{Definition \thetcbcounter \ifthenelse{\isempty{#1}}{}{: #1} \\ },
    sharp corners
}

\newcommand*{\eqbox}{\tcboxmath[
    enhanced,
    colback=black!10!white,
    colframe=black,
    sharp corners,
    size=fbox,
    boxsep=8pt,
    boxrule=1pt
]}

\newtcolorbox[auto counter, number within=chapter]{exbox}{
    parbox=false,
    breakable,
    enhanced,
    sharp corners,
    boxrule=1pt,
    colback=white,
    colframe=black,
    before upper= \textbf{Example \thetcbcounter:}\,,
    before lower= \textbf{Solution:}\,,
    segmentation hidden
}

\newtcolorbox{resbox}{
    enhanced,
    colback=black!10!white,
    colframe=black,
    boxrule=1pt,
    boxsep=0pt,
    top=2pt,
    ams nodisplayskip,
    sharp corners
}


\begin{document}

\prob{1}{
Calculate the series
\begin{eqnarray}
   I_1 = \sum_{n=1}^{\infty} \frac{1}{4^{n}}, \quad I_2 = \sum_{n=0}^{\infty} \frac{(-1)^{n}}{3^{n}}
.\end{eqnarray}
}

\sol{
Recall that for a geometric series $\sum_{n=0}^{\infty} a r^{n} = a/(1 - r)$.

For $I_1$, we can write
\begin{eqnarray}
    \eqbox{I_1 = \sum_{n=0}^{\infty} \frac{1}{4^{n}} - 1 = \frac{1}{1 - 1/4} - 1 = \frac{1}{3}}
.\end{eqnarray}

For $I_2$, $a = 1$ and $r = -1/3$, meaning $I_{2}$
\begin{eqnarray}
    \eqbox{I_{2} = \frac{1}{1 + 1/3} = \frac{3}{4}}
.\end{eqnarray}
}


\prob{2}{
Calculate the sum
\begin{eqnarray}
    I_3 = \sum_{n=1}^{N} \Big[ \frac{1}{n} - \frac{1}{1+n} \Big]
.\end{eqnarray}
What is the limit of $I_3$ at $N \rightarrow \infty$.
}

\sol{
We can split $I_3$ into two series as
\begin{eqnarray}
   I_3 = \sum_{n=1}^{N} \frac{1}{n} - \sum_{n=1}^{N} \frac{1}{n+1} 
,\end{eqnarray}
and we can shift indices on the second series with the substitution $m = n+1$ such that
\begin{eqnarray}
    \eqbox{I_3 = \sum_{n=1}^{N} \frac{1}{n} - \sum_{m=2}^{N+1} \frac{1}{m} = 1 - \frac{1}{N+1} = \frac{N}{N+1}}
.\end{eqnarray}
The infinite series
\begin{eqnarray}
    \eqbox{\lim_{N \rightarrow \infty} I_3 = \sum_{n=1}^{\infty} \Big[ \frac{1}{n} - \frac{1}{n+1} \Big] = 1}
\end{eqnarray}
since $1/(N+1) \rightarrow 0$ as $N \rightarrow \infty$
}


\prob{3}{
Does the alternating series
\begin{eqnarray}
   I_1 = \sum_{n=1}^{\infty} \frac{(-1)^{n}}{n^{7/6}}
\end{eqnarray}
converge absolutely?
}

\sol{
The series $I_1$ converges absolutely:
\begin{eqnarray}
    \sum_{n=1}^{\infty} \Big| \frac{(-1)^{n}}{n^{7/6}} \Big| = \sum_{n=1}^{\infty} \frac{1}{n^{7/6}} < \int_{1}^{\infty} \frac{\dd{x}}{x^{7/6}} = -6 x^{-1/6} \big|_{1}^{\infty} = 6
.\end{eqnarray}
Since the series is bounded above and below $I_1$ is absolutely convergent.
}


\prob{4}{
Does the series
\begin{eqnarray}
    I_2 = \sum_{n=1}^{\infty} \frac{x}{(nx + 1)[(n + 1)x + 1]}
\end{eqnarray}
with $x > 0$ converge uniformly?
}

\sol{

Notice that we can write the series as
\begin{eqnarray}
    I_{2} = \sum_{n=1}^{\infty} \Big[ \frac{1}{nx + 1} - \frac{1}{(n+1)x + 1} \Big] = \sum_{n=1}^{\infty} \frac{1}{nx + 1} - \sum_{n=1}^{\infty} \frac{1}{(n+1)x + 1}
,\end{eqnarray}
The $N^{\rm th}$ partial sum is given as
\begin{eqnarray}
    I_{2,N} = \sum_{n=1}^{N} \frac{1}{nx + 1} - \sum_{n=1}^{N} \frac{1}{(n+1)x + 1}
,\end{eqnarray}
and the second sum can be rewritten by shifting indices to $m = n+1$ such that it is clear that the partial sum is simply
\begin{eqnarray}
   I_{2,N} = \frac{1}{x + 1} - \frac{1}{(N+1)x + 1}
.\end{eqnarray}
Taking the limit $N \rightarrow \infty$, we find
\begin{eqnarray}
   \eqbox{I_{2} = \frac{1}{x+1}} 
.\end{eqnarray}
which is continuous on $x > 0$, implying that $I_{2}$ is uniformly convergent on $x > 0$.
}


\prob{5}{
Calculate the infinite product
\begin{eqnarray}
    I_3 = \prod_{n=1}^{\infty} \Big[ 1 + \frac{(-1)^{n}}{n+1} \Big]
.\end{eqnarray}

}

\sol{

We can heuristically solve this problem by writing the first several terms of the product:
\begin{eqnarray}
\begin{aligned}
    I_{3} &= \Big( 1 - \frac{1}{2} \Big) \Big( 1 + \frac{1}{3} \Big) \Big( 1 - \frac{1}{4} \Big) \Big( 1 + \frac{1}{5} \Big) \Big( 1 - \frac{1}{6} \Big) \cdots \\
          &= \frac{1}{2} \cdot \frac{4}{3} \cdot \frac{3}{4} \cdot \frac{6}{5} \cdot \frac{5}{6} \cdots = \eqbox{ \frac{1}{2} }
.\end{aligned}
\end{eqnarray}

More formally, we can write the product in terms of its even and odd terms as follows
\begin{eqnarray}
\begin{aligned}
    I_{3} &= \frac{1}{2} \prod_{k=1}^{\infty} \Big[ 1 + \frac{(-1)^{2k}}{2k + 1} \Big] \Big[ 1 + \frac{(-1)^{2k+1}}{(2k+1) + 1} \Big] \\
          &= \frac{1}{2} \prod_{k=1}^{\infty} \frac{2k + 2}{2k + 1} \cdot \frac{2k + 1}{2k + 2} = \frac{1}{2}
.\end{aligned}
\end{eqnarray}

}




\end{document}
