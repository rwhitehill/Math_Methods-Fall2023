\def\duedate{\today}
\def\HWnum{9}
\documentclass[10pt,a4paper]{book}

% custom section formatting
\usepackage{titlesec}
\titleformat{\chapter}[display]
{\normalfont\Large\filcenter\sffamily}
{\titlerule[1pt]%
\vspace{1pt}%
\titlerule
\vspace{1pc}%
\LARGE\MakeUppercase{\chaptertitlename} \thechapter}
{1pc}
{\titlerule
\vspace{1pc}%
\Huge}

% appendix handling
\usepackage[toc,page]{appendix}
    
% encoding for file and font
\usepackage[utf8]{inputenc}
\usepackage[T1]{fontenc}

% math formatting/tools
\usepackage{amsmath}
\usepackage{amssymb}
\usepackage{mathtools}
\usepackage[arrowdel]{physics}
\usepackage{dsfont}

\newcommand{\R}{\mathbb{R}}
\newcommand{\Z}{\mathbb{Z}}
\newcommand{\N}{\mathbb{N}}
\newcommand{\Q}{\mathbb{Q}}
\newcommand{\C}{\mathbb{C}}

% unit formatting
\usepackage{siunitx}
\AtBeginDocument{\RenewCommandCopy\qty\SI}

% figure formatting/tools
\usepackage{graphicx}
\usepackage{float}
\usepackage{subcaption}
\usepackage{multirow}
\usepackage{import}
\usepackage{pdfpages}
\usepackage{transparent}
\usepackage{currfile}

\NewDocumentCommand\incfig{O{1} m}{
    \def\svgwidth{#1\textwidth}
    \import{./Figures/\currfiledir}{#2.pdf_tex}
}

\newcommand{\bef}{\begin{figure}[h!tb]\centering}
\newcommand{\eef}{\end{figure}}

\newcommand{\bet}{\begin{table}[h!tb]\centering}
\newcommand{\eet}{\end{table}}

% hyperlink references 
\usepackage{hyperref}
\hypersetup{
    colorlinks=true,
    linkcolor=blue,
    filecolor=magenta,
    urlcolor=cyan,
    pdftitle={Physics 1 Notes},
    pdfauthor={Richard Whitehill},
    pdfpagemode=FullScreen
}
\urlstyle{same}

\newcommand{\eref}[1]{Eq.~(\ref{eq:#1})}
\newcommand{\erefs}[2]{Eqs.~(\ref{eq:#1})--(\ref{eq:#2})}

\newcommand{\fref}[1]{Fig.~(\ref{fig:#1})}
\newcommand{\frefs}[2]{Fig.~(\ref{fig:#1})--(\ref{fig:#2})}

\newcommand{\aref}[1]{Appendix~(\ref{app:#1})}
\newcommand{\sref}[1]{Section~(\ref{sec:#1})}
\newcommand{\srefs}[2]{Sections~(\ref{sec:#1})-(\ref{sec:#2})}

\newcommand{\tref}[1]{Table~(\ref{tab:#1})}
\newcommand{\trefs}[2]{Table~(\ref{tab:#1})--(\ref{tab:#2})}

% tcolorbox formatting/definitions
\usepackage[most]{tcolorbox}
\usepackage{xcolor}
\usepackage{xifthen}
\usepackage{parskip}

\definecolor{peach}{rgb}{1.0,0.8,0.64}

\DeclareTColorBox[auto counter, number within=chapter]{defbox}{O{}}{
    enhanced,
    boxrule=0pt,
    frame hidden,
    borderline west={4pt}{0pt}{green!50!black},
    colback=green!5,
    before upper=\textbf{Definition \thetcbcounter \ifthenelse{\isempty{#1}}{}{: #1} \\ },
    sharp corners
}

\newcommand*{\eqbox}{\tcboxmath[
    enhanced,
    colback=black!10!white,
    colframe=black,
    sharp corners,
    size=fbox,
    boxsep=8pt,
    boxrule=1pt
]}

\newtcolorbox[auto counter, number within=chapter]{exbox}{
    parbox=false,
    breakable,
    enhanced,
    sharp corners,
    boxrule=1pt,
    colback=white,
    colframe=black,
    before upper= \textbf{Example \thetcbcounter:}\,,
    before lower= \textbf{Solution:}\,,
    segmentation hidden
}

\newtcolorbox{resbox}{
    enhanced,
    colback=black!10!white,
    colframe=black,
    boxrule=1pt,
    boxsep=0pt,
    top=2pt,
    ams nodisplayskip,
    sharp corners
}


\begin{document}

\prob{1}{

For a region of volume $V$, calculate the following integrals taken over the surface of this region:
\begin{eqnarray}
    \va*{I}_1 = \oint_{S} \va*{r}(\va*{a} \cdot \vu*{n}) \dd{S}, \quad \va*{I}_2 = \oint_{S} (\va*{a} \cdot \va*{r}) \vu*{n} \dd{S}
,\end{eqnarray}
where $\vu*{n}(\va*{r})$ is a local unit vector normal to the surface and $\va*{a}$ is a constnat vector.

HINT: Try to multiply $I_1$ and $I_2$ by an auxiliary constant vector $\va*{b}$.

}

\sol{

For the first integral notice that
\begin{eqnarray}
\begin{aligned}
    \va*{I}_1 \cdot \va*{b} &= \oint_{S} (\va*{b} \cdot \va*{r}) \va*{a} \cdot \vu*{n} \dd{S} \\
                            &= \int_{V} \div{ (\va*{b} \cdot \va*{r}) \va*{a} } \, \dd{V} = \int_{V} \va*{a} \cdot \grad{ (\va*{b} \cdot \va*{r}) } \, \dd{V} \\
                            &= \int_{V} \va*{a} \cdot [ (\va*{b} \cdot \grad) \va*{r} + \va*{b} \cross ( \curl{\va*{r}} ) ] \dd{V} = (\va*{b} \cdot \va*{a}) V
.\end{aligned}
\end{eqnarray}
Thus, we have
\begin{eqnarray}
    \va*{I}_{1} = V \va*{a}
\end{eqnarray}
for any generic volume $V$ with bounding surface $S$.

Similarly, for the second integration,
\begin{eqnarray}
    \va*{I}_2 &= \oint_{S} (\va*{a} \cdot \va*{r}) \va*{b} \cdot \vu*{n} \dd{S} = \int_{V} \div{ (\va*{a} \cdot \va*{r})\va*{b} } = V \va*{a}
.\end{eqnarray}

}


\prob{2}{

Calculate  the following determinant:
\begin{eqnarray}
    \begin{vmatrix}
        E & V & V \\
        V & E & V \\
        V & V & E
    \end{vmatrix}
.\end{eqnarray}
Find $E$ at which the determinant vanishes.
Solution of this problem determines the energy levels of an electron in a triangular molecule.

}

\sol{

The determinant of the matrix above is just
\begin{eqnarray}
    E( E^2 - V^2 ) + V( V^2 - EV ) + V( EV - V^2 ) = E(E^2 - V^2)
.\end{eqnarray}
If we require this be zero, then it is clear that the solutions are $E = -V,0,V$.

}


\prob{3}{

Calculate the inverese matrix $A^{-1}$ for 
\begin{eqnarray}
    A = \begin{pmatrix}
        1 & 2 & 3 \\
        2 & 3 & 1 \\
        3 & 1 & 2
    \end{pmatrix}
.\end{eqnarray}

}

\sol{

We will solve this by using the cofactor method, which is somewhat tedious already at $3 \times 3$ but also the most direct.
First, we have $\det(A) = -18$.
Next, the cofactor matrix is just
\begin{eqnarray}
    C = \begin{pmatrix}
        5 & -1 & -7 \\
        -1 & -7 & 5 \\
        -7 & 5 & -1
    \end{pmatrix}
.\end{eqnarray}
The inverse matrix is then just
\begin{eqnarray}
    A^{-1} = \frac{1}{18} \begin{pmatrix}
        -5 & 1 & 7 \\
        1 & 7 & -5 \\
        7 & -5 & 1
    \end{pmatrix}
.\end{eqnarray}

}


\prob{4}{

Consider two Hermitian matrices:
\begin{eqnarray}
    A = \begin{pmatrix}
        a & 0 & 0 \\
        0 & b & 0 \\
        0 & 0 & c
    \end{pmatrix}
    ,\quad
    B = \begin{pmatrix}
        0 & iv & 0 \\
        -iv & 0 & 0 \\
        0 & 0 & u
    \end{pmatrix}
,\end{eqnarray}
where $a,~b,~c,~u~,{\rm and}~v$ are real constants.

(a) Is the product $AB$ a Hermitian matrix?

(b) Do $A$ and $B$ commute?

(c) What are the relations between $a,b,c,u,v$ for which the answer to the questions $1$ and $2$ is yes?

}

\sol{

(a) It is a general fact that if $A$ and $B$ are hermitian, then their product $AB$ is also Hermitian: $(AB)^{\dagger}(AB) = B^{\dagger} A^{\dagger} A B = B^{\dagger} B = I$.

(b) It is easy to see that $A$ and $B$ do not commute in general:
\begin{eqnarray}
    AB = \begin{pmatrix}
        0 & iav & 0 \\
        -ibv & 0 & 0 \\
        0 & 0 & cu
    \end{pmatrix}
    , \quad
    BA = \begin{pmatrix}
        0 & ibv & 0 \\
        -iav & 0 & 0 \\
        0 & 0 & cu
    \end{pmatrix}
.\end{eqnarray}

(c) As stated in part (a), these product of these Hermitian matrices is Hermitian for any $a,b,c,u,v$.
In part (b), though, we saw that the products $AB$ and $BA$ do not generally commute with each other.
However, if $a = -b$, then $[A,B] = 0$.

}


\prob{5}{

Find the eigenvalues and eigenvectors of this matrix:
\begin{eqnarray}
    A = \begin{pmatrix}
        1 & 0 & 0 \\
        0 & 0 & i \\
        0 & -i & 1
    \end{pmatrix}
.\end{eqnarray}

}

\sol{

By inspection, we can see that $1$ is an eigenvalue corresponding to the eigenvector $\begin{pmatrix} 1 & 0 & 0 \end{pmatrix}^{T}$.
We therefore only have to solve for the eigenvalues and eigenvectors of the square matrix in the lower right block of $A$:
\begin{eqnarray}
    \begin{vmatrix}
        -\lambda & i \\
        -i & 1-\lambda
    \end{vmatrix}
    = \lambda(\lambda - 1) - 1 = \lambda^2 - \lambda - 1 = 0
.\end{eqnarray}
This gives the other two eigenvalues as
\begin{eqnarray}
    \lambda_{\pm} = \frac{1 \pm \sqrt{5}}{2}
.\end{eqnarray}
We then can solve for the other two eigenvectors (up to a constant) by imposing
\begin{eqnarray}
    -\lambda_{\pm} x_1 + i x_2 = 0 \Rightarrow x_2 = -i \lambda_{\pm} x_1
.\end{eqnarray}
Thus, the eigenvector
\begin{eqnarray}
    x_{\pm} = \begin{pmatrix}
    1 \\ -i \lambda_{\pm}
    \end{pmatrix}
.\end{eqnarray}

}



\end{document}
