\def\duedate{\today}
\def\HWnum{3}
\input{../preamble.tex}

\begin{document}

\prob{1}{

Express the following delta function in terms of delta functions of the variable $x$:
\begin{eqnarray}
    \delta\Big( \frac{\sin{x}}{x} \Big) 
.\end{eqnarray}

}

\sol{

Recall that we can write 
\begin{eqnarray}
    \delta(f(x)) = \sum_{i} \frac{\delta(x - x_{i})}{|f'(x_{i})|}
,\end{eqnarray}
where $f(x)$ has simple roots $x_{i}$\footnote{Otherwise the $f'(x_{i})$ term vanishes and we have to be a bit more careful with the expansion of $f(x)$ inside the delta function.}.
Note that $f(x) = \sin{x}/x$ indeed has simple roots since we can write
\begin{eqnarray}
    \frac{\sin{x}}{x} = \prod_{n=1}^{\infty} \Big( 1 - \frac{x^2}{n^2 \pi^2} \Big)
.\end{eqnarray}
From this formala, it is also trivial to glean that $\sin{x}/x$ has zeros when $\sin{x} = 0$ or $x = n \pi$ for $n = \pm 1, \pm 2, \ldots$.
Additionally, $\dv{x} \sin{x}/x |_{x = n \pi}= [\cos{x}/x - \sin{x}/x^2]_{x = n\pi} = (-1)^{n}/n \pi$.

Thus,
\begin{eqnarray}
    \eqbox{ \delta \Big( \frac{\sin{x}}{x} \Big) = \sum_{n=1}^{\infty} n \pi \Big[ \delta(x - n \pi) + \delta(x + n\pi) \Big] }
.\end{eqnarray}


}


\prob{2}{

Calculate
\begin{eqnarray}
    I(z) = \Gamma(1 + z) \Gamma(1 - z)
\end{eqnarray}
at $z = 1/4$.

}

\sol{

We can write 
\begin{eqnarray}
    \eqbox{ 
\begin{aligned}
    &\Gamma(1 + z)\Gamma(1 - z) = z \Gamma(z) \Gamma(1 - z) = z \frac{\pi}{\sin{\pi z}} \\
                               &\Rightarrow \Gamma(1 + z)\Gamma(1 - z)|_{z = 1/4} = \frac{\pi/4}{\sin{\pi/4}} = \frac{\pi}{2 \sqrt{2}} 
\end{aligned}
}
.\end{eqnarray}

}


\prob{3}{

    Using the definition of the complete elliptical integrals $E(m)$ and $K(m)$, express the derivative ${\partial E(m)}/{\partial m}$ in terms of $K(m)$ and $E(m)$.

}

\sol{

Recall that the complete elliptic integrals of the first and second kind are defined as
\begin{align}
    K(m) &= \int_{0}^{\pi/2} \frac{\dd{\theta}}{\sqrt{1 - m\sin^2{\theta}}} \\
    E(m) &= \int_{0}^{\pi/2} \dd{\theta} \sqrt{1 - m \sin^2{\theta}}
,\end{align}
respectively.
Differentiating $E(m)$ with respect to $m$ we have
\begin{eqnarray}
    \begin{aligned}
        \pdv{E}{m} &= \pdv{m} \int_{0}^{\pi/2} \dd{\theta} \sqrt{1 - m\sin^2{\theta}} = \int_{0}^{\pi/2} \dd{\theta} \pdv{m} \sqrt{1 - m \sin^2{\theta}} \\
                   &= -\frac{1}{2} \int_{0}^{\pi/2} \frac{\sin^2{\theta} \dd{\theta}}{\sqrt{1 - m \sin^2{\theta}}}
    .\end{aligned}
\end{eqnarray}
Observe the following:
\begin{eqnarray}
    \begin{aligned}
        K(m) - E(m) &= \int_{0}^{\pi/2} \dd{\theta} \Bigg[ \frac{1}{\sqrt{1 - m\sin^2{\theta}}} - \sqrt{1 - m\sin^2{\theta}} \Bigg] \\ 
                    &= \int_{0}^{\pi/2} \dd{\theta} \frac{1 - (1 - m \sin^2{\theta})}{\sqrt{1 - m\sin^2{\theta}}} = m \int_{0}^{\pi/2} \dd{\theta} \frac{\sin^2{\theta}}{\sqrt{1 - m \sin^2{\theta}}}
    .\end{aligned}
\end{eqnarray}
Thus,
\begin{eqnarray}
    \eqbox{ \pdv{E}{m} = \frac{K(m) - E(m)}{m} } 
.\end{eqnarray}

{
\color{red}
The final result above is not correct.
It should be
\begin{eqnarray}
    \pdv{E}{m} = -\frac{1}{2} \frac{K(m) - E(m)}{m} = \frac{E(m) - K(m)}{2m}
.\end{eqnarray}

}

}

\prob{4}{

    Find the values of $e^{\pm i\pi/2}$, $e^{i \pi n}$, $\ln(-1)$ where $n = 0, \pm 1, \pm 2, \ldots$.

}

\sol{

We find that
\begin{gather}
    \eqbox{ e^{\pm i \pi/2} = \cos{ (\pm \pi/2) } + i \sin{( \pm \pi/2 )} = \pm i } \\
    \eqbox{ e^{i \pi n} = \cos{(\pi n)} + i \sin{(\pi n)} = (-1)^{n} } \\
    \eqbox{ \ln(-1) = \ln(e^{i\pi}) = i \pi }
,\end{gather}
restricting our definition of $\ln{z}$ to the principal branch (i.e. $\arg{z} \in [0,2 \pi)$).

}


\prob{5}{

Calculate the following series:
\begin{eqnarray}
    I_1 = \sum_{n=0}^{\infty} p^{n} \sin{(qn)} \quad {\rm and} \quad I_{2} = \sum_{n=0}^{\infty} p^{n} \cos{(qn)}
,\end{eqnarray}
where $p$ and $q$ are real parameters. 

HINT: Use the sum of geometric series with complex $r$.

}

\sol{

Observe the following:
\begin{eqnarray}
   I = \sum_{n=0}^{\infty} p^{n} e^{iqn} = I_{2} + i I_{1}
.\end{eqnarray}
That is $I_{2} = \Re(I)$ and $I_{1} = \Im(I)$.
We can use the geometric series formula with complex $r = p e^{iq}$, giving
\begin{eqnarray}
    I = \frac{1}{1 - p e^{iq}} = \frac{1}{(1 - p \cos{q}) - i p \sin{q}} = \frac{(1 - p\cos{q}) + i p \sin{q}}{(1 - p\cos{q})^2 + p^2 \sin^2{q}}
.\end{eqnarray}
Taking real and imaginary parts of $I$, we have
\begin{gather}
    \eqbox{ I_{1} = \frac{p \sin{q}}{ 1 + p^2 - 2 p \cos{q}} } \\
    \eqbox{ I_{2} = \frac{1 - p\cos{q}}{1 + p^2 - 2p \cos{q}} }
.\end{gather}



}



\end{document}
