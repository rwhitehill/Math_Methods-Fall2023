\def\duedate{\today}
\def\HWnum{2}
\documentclass[10pt,a4paper]{book}

% custom section formatting
\usepackage{titlesec}
\titleformat{\chapter}[display]
{\normalfont\Large\filcenter\sffamily}
{\titlerule[1pt]%
\vspace{1pt}%
\titlerule
\vspace{1pc}%
\LARGE\MakeUppercase{\chaptertitlename} \thechapter}
{1pc}
{\titlerule
\vspace{1pc}%
\Huge}

% appendix handling
\usepackage[toc,page]{appendix}
    
% encoding for file and font
\usepackage[utf8]{inputenc}
\usepackage[T1]{fontenc}

% math formatting/tools
\usepackage{amsmath}
\usepackage{amssymb}
\usepackage{mathtools}
\usepackage[arrowdel]{physics}
\usepackage{dsfont}

\newcommand{\R}{\mathbb{R}}
\newcommand{\Z}{\mathbb{Z}}
\newcommand{\N}{\mathbb{N}}
\newcommand{\Q}{\mathbb{Q}}
\newcommand{\C}{\mathbb{C}}

% unit formatting
\usepackage{siunitx}
\AtBeginDocument{\RenewCommandCopy\qty\SI}

% figure formatting/tools
\usepackage{graphicx}
\usepackage{float}
\usepackage{subcaption}
\usepackage{multirow}
\usepackage{import}
\usepackage{pdfpages}
\usepackage{transparent}
\usepackage{currfile}

\NewDocumentCommand\incfig{O{1} m}{
    \def\svgwidth{#1\textwidth}
    \import{./Figures/\currfiledir}{#2.pdf_tex}
}

\newcommand{\bef}{\begin{figure}[h!tb]\centering}
\newcommand{\eef}{\end{figure}}

\newcommand{\bet}{\begin{table}[h!tb]\centering}
\newcommand{\eet}{\end{table}}

% hyperlink references 
\usepackage{hyperref}
\hypersetup{
    colorlinks=true,
    linkcolor=blue,
    filecolor=magenta,
    urlcolor=cyan,
    pdftitle={Physics 1 Notes},
    pdfauthor={Richard Whitehill},
    pdfpagemode=FullScreen
}
\urlstyle{same}

\newcommand{\eref}[1]{Eq.~(\ref{eq:#1})}
\newcommand{\erefs}[2]{Eqs.~(\ref{eq:#1})--(\ref{eq:#2})}

\newcommand{\fref}[1]{Fig.~(\ref{fig:#1})}
\newcommand{\frefs}[2]{Fig.~(\ref{fig:#1})--(\ref{fig:#2})}

\newcommand{\aref}[1]{Appendix~(\ref{app:#1})}
\newcommand{\sref}[1]{Section~(\ref{sec:#1})}
\newcommand{\srefs}[2]{Sections~(\ref{sec:#1})-(\ref{sec:#2})}

\newcommand{\tref}[1]{Table~(\ref{tab:#1})}
\newcommand{\trefs}[2]{Table~(\ref{tab:#1})--(\ref{tab:#2})}

% tcolorbox formatting/definitions
\usepackage[most]{tcolorbox}
\usepackage{xcolor}
\usepackage{xifthen}
\usepackage{parskip}

\definecolor{peach}{rgb}{1.0,0.8,0.64}

\DeclareTColorBox[auto counter, number within=chapter]{defbox}{O{}}{
    enhanced,
    boxrule=0pt,
    frame hidden,
    borderline west={4pt}{0pt}{green!50!black},
    colback=green!5,
    before upper=\textbf{Definition \thetcbcounter \ifthenelse{\isempty{#1}}{}{: #1} \\ },
    sharp corners
}

\newcommand*{\eqbox}{\tcboxmath[
    enhanced,
    colback=black!10!white,
    colframe=black,
    sharp corners,
    size=fbox,
    boxsep=8pt,
    boxrule=1pt
]}

\newtcolorbox[auto counter, number within=chapter]{exbox}{
    parbox=false,
    breakable,
    enhanced,
    sharp corners,
    boxrule=1pt,
    colback=white,
    colframe=black,
    before upper= \textbf{Example \thetcbcounter:}\,,
    before lower= \textbf{Solution:}\,,
    segmentation hidden
}

\newtcolorbox{resbox}{
    enhanced,
    colback=black!10!white,
    colframe=black,
    boxrule=1pt,
    boxsep=0pt,
    top=2pt,
    ams nodisplayskip,
    sharp corners
}


\begin{document}

\prob{1}{

Calculate the following limit:
\begin{eqnarray}
    I_{1} = \lim_{x \rightarrow 0} \frac{x^2 \ln(1 + x^2)}{x^2 - \sin^2{x}}
.\end{eqnarray}

}

\sol{

Upon first glance, we have a limit of an indeterminate form $0/0$.
We may use L'Hopital's rule, but the functions are quite messy to differentiate, so we expand both the numerator and denominator in Taylor series about $x = 0$.
Note that $\ln(1 + x) = x - x^2/2 + {\rm cal O}(x^3)$ and $\sin^2{x} = (1 - \cos{2x})/2 = [(2x)^2/2! + (2x)^{4}/4! + {\cal O}(x^{6})]/2$.
Thus, the limit
\begin{eqnarray}
    \eqbox{ I_{1} = \lim_{x \rightarrow 0} \frac{ x^{4} }{ x^2 - \frac{1}{2}[(2x)^2/2! - (2x)^{4}/4!] } = \frac{2(4!)}{2^{4}} = \frac{24}{8} = 3 }
.\end{eqnarray}


}


\prob{2}{

Calculate the following limit of the $m-{\rm th}$ derivative at $x = 0$:
\begin{eqnarray}
    I_{2} = \lim_{x \rightarrow 0} \dv[m]{x} \frac{\ln(1 + x) - x}{x^2}
.\end{eqnarray}

}

\sol{

Here, we can write
\begin{eqnarray}
    \begin{aligned}
        \frac{\ln(1 + x) - x}{x^2} &= \frac{1}{x^2} \Big[ \sum_{n=1}^{\infty} \frac{(-1)^{n+1} x^{n}}{n} - x \Big] = \sum_{n=2}^{\infty} \frac{(-1)^{n+1}}{n} x^{n-2} = \sum_{n=0}^{\infty} \frac{(-1)^{n+1}}{n+2} x^{n} \\
                                   &= \sum_{n=0}^{m-1} \frac{(-1)^{n+1}}{n+2} x^{n} + \frac{(-1)^{m+1}}{m+2} x^{m} + \sum_{n=m+1} \frac{(-1)^{n+1}}{n+2} x^{n}
    .\end{aligned}
\end{eqnarray}
The last step is mostly illustrative for this:
\begin{eqnarray}
    \dv[m]{x} \frac{\ln(1 + x) - x}{x^2} = \frac{(-1)^{m+1} \, m!}{m+2} + {\cal O}(x)
.\end{eqnarray}
Therefore, taking $x \rightarrow 0$ gives
\begin{eqnarray}
    \eqbox{ I_{2} = (-1)^{m+1} \frac{m!}{m+2} }
.\end{eqnarray}


}


\prob{3}{

Calculate the following limit at $x = y = 0$:
\begin{eqnarray}
    I_{3} = \lim_{x \rightarrow 0, y \rightarrow 0} \laplacian \big[ e^{-ax^2 - by^2} \cos{ax} \cos{by} \big]
.\end{eqnarray}

}

\sol{

We can rewrite the operand of the laplacian as
\begin{eqnarray}
    e^{-ax^2-by^2} \cos{ax} \cos{by} = [e^{-ax^2}\cos{ax}] [e^{-by^2} \cos{bx}]
,\end{eqnarray}
which gives
\begin{eqnarray}
    \laplacian e^{-ax^2 - by^2} \cos{ax} \cos{by} = e^{-by^2} \cos{by} \pdv[2]{x} e^{-ax^2} \cos{ax} + e^{-ax^2} \cos{ax} \pdv[2]{y} e^{-by^2} \cos{by}
.\end{eqnarray}
Note that we can evaluate the limit of the first term and use the replacements $a \leftrightarrow b,~ x \leftrightarrow y$.

Observe that
\begin{eqnarray}
    \lim_{y \rightarrow 0} e^{-b y^2} \cos{b y} = 1
.\end{eqnarray}
Next, we evaluate the second factor to be
\begin{eqnarray}
    \begin{aligned}
        \dv[2]{x} e^{-ax^2} \cos{ax} &= -2a (1 - 2ax) e^{-ax^2} \cos{ax} + (- 2 a x e^{-ax^2})(-a \sin{ax}) \\
                                     &+ e^{-a x^2} (-a^2 \cos{ax})
    .\end{aligned}
\end{eqnarray}
It should be clear then that
\begin{eqnarray}
    \lim_{x \rightarrow 0} \pdv[2]{x} e^{-ax^2} \cos{ax} = -2a - a^2 = -(a + 1)^2 + 1 
.\end{eqnarray}
Hence 
\begin{eqnarray}
    \eqbox{ \lim_{x,y \rightarrow 0} \laplacian e^{-ax^2 - by^2} \cos{ax} \cos{by} = -(a + 1)^2 - (b + 1)^2 + 2 }
.\end{eqnarray}



}


\prob{4}{

Calculate the sum
\begin{eqnarray}
   I_{1} = \sum_{n=1}^{N} (n^2 + n + 1) 
.\end{eqnarray}

}


\prob{5}{

Calculate the sum
\begin{eqnarray}
   I_2 = \sum_{n=1}^{\infty} \frac{(-1)^{n}}{n^2 + a^2} 
.\end{eqnarray}

}


\prob{6}{

Calculate the asymptotic series for this integral at $x \gg 1$:
\begin{eqnarray}
    I_{3} = \int_{x}^{\infty} e^{-au} \ln{u} \dd{u}
,\end{eqnarray}
where $a > 0$.

}

\sol{

Observe that we can write
\begin{eqnarray}
    \begin{aligned}
    I_{3} &= \int_{x}^{\infty} e^{-au} (\ln{au} - \ln{a}) \dd{u} = \int_{x}^{\infty} e^{-au} \ln{au} - \ln{a} \int_{x}^{\infty} e^{-au} \dd{u} \\
          &= -\frac{1}{a} \int_{x}^{\infty} e^{-v} \ln{v} \dd{v} - \frac{\ln{a}}{a} e^{-ax}
    \end{aligned}
.\end{eqnarray}

We can solve this problem using integration by parts $\int f' g \dd{x} = fg - \int f g' \dd{x}$ with $f' = e^{-au}$ and $g = \ln{u}$, which gives $f = -\frac{1}{a}e^{-au}$ and $g' = 1/u$

}



\end{document}
