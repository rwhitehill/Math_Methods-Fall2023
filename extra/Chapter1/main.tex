\chapter{Sturm-Liouville Differential Equation}

\section{Motivation}

Many of the problems we have studied follow from general properties of the Sturm-Lioville (SL) problem.
In these problems, a system is governed by the differential equation of the form 
\begin{eqnarray}
    \label{eq:SL-DE}
    -\dv{x}\Bigg[ p(x) \dv{x} \Bigg] \phi(x) + q(x) \phi(x) =  \lambda r(x) \phi(x)
\end{eqnarray}
for $x \in [a,b]$, where $a,b \in \R$ and some real functions $p,q,r$, where $p(x),r(x) > 0$ for $x \in (a,b)$\footnote{Note that the endpoints are excluded here!}.
This is in essence an eigenvalue problem, where we would like to determine the eigenvalues $\lambda$ and corresponding eigenfunctions $\phi(x)$ which satisfy \eref{SL-DE} for a given set of $p,q,r$.

For example, the Schr\"{o}dinger equation, Bessel's equation, and Legendre's equation all fall under this category.
Clearly, the particular properties that determines a system's unique behavior depend on $p,q,r$, but the general methods by which we uncover these all follow the same generic trend which follow from general behaviors of systems with a Sturm-Lioville description.


\section{Boundary Conditions}

Since the \eref{SL-DE} is a second-order differential equation, we have two linearly independent solutions $y_{1,2}$ for a given set of functions $p,q,r$, and a general solution $y = \alpha_1 y_1 + \alpha_2 y_2$, where $\alpha_{1,2}$ are constants.
These constants are determined by boundary conditions at $x=a,b$ by specifying (1) $y(a)$ and $y(b)$ [Dirichlet BCs], (2) $y'(a)$ and $y'(b)$ [Neumann BCs], or (3) $c_a y(a) + d_{a} y'(a)$ and $c_{b} y(b) + d_{b}y'(b)$ [Robin BCs].
For the most part, we focus on Dirichlet BCs (e.g. specifying the value of the wave function) and Neumann (e.g. specifying the surface charge density).
It is rare in the textbook problems that a linear combination of the function and its derivative are specified since these linear combinations are not usually related to a physical quantity.
For the development of general properties, though, we will reference these kind of BCs, but Dirichlet and Neumann BCs are recovered by setting $d_{a,b} = 0$ and $c_{a,b} = 0$, respectively.

Finally, in some cases, a fourth distinct set of BCs can be specified, which are called periodic BCs.
As the name suggests we either have $y(a) = y(b)$ or $y'(a) = y'(b)$.

\section{Definitions}

Let us define a couple of terms that will be used to identify the type of equation we consider.

\begin{enumerate}
    \item A \textit{regular} SL system is one such that homogeneous mixed BCs are given: $c_{a}y(a) + d_{a}y'(a) = 0$ and $c_{b}y(b) + d_{b}y'(b) = 0$
    \item A \textit{periodic} SL system is one such that periodic BCs are specified and $p(a) = p(b)$
    \item A \textit{singular} SL system is one where any of the following occur:
    \begin{itemize}
        \item $p(a) = 0$, no BC at $a$ is given, and the BC at $b$ is homogeneous mixed (Note: solutions must be bounded at $x=a$)\footnote{A function $f$ is bounded at $x$ if $|f(x)| < M$ for some $M$.}
        \item $p(b) = 0$, no BC at $b$ is given, and the BC at $a$ is homogeneous mixed (Note: solutions must be bounded at $x = a$)
        \item $p(a) = p(b) = 0$ and no BCs are given (solutions must be bounded at both $x = a,b$.)
        \item $a \rightarrow -\infty$ and $b \rightarrow \infty$ such that the equation is defined on $\R$ (Note: solutions must be square-integrable on $\R$)\footnote{A function $f$ is square-integrable if $\int_{-\infty}^{\infty} |f(x)|^2 < \infty$ (i.e. the integral is convergent and bounded).}
    \end{itemize}
\end{enumerate}


\section{Properties of the Sturm-Liouville System}

\subsection{Sturm-Lioville Operator}

Let $\mathcal{L}^2([a,b],r(x),\dd{x})$ be the Hilbert space of square integrable functions on the interval $[a,b]$ with inner product
\begin{eqnarray}
    (f,g) = \int_{a}^{b} r(x) f^{*}(x) g(x) \dd{x}
.\end{eqnarray}
It is for this reason that $r(x)$ is sometimes denoted a \textit{weight} function.

Denote the linear differential operator
\begin{eqnarray}
    \label{eq:SL-lin-op}
    \hat{L} = \frac{1}{r(x)} \Bigg[ - \dv{x} p(x) \dv{x} + q(x) \Bigg]
,\end{eqnarray}
and let $\mathcal{H} \subset \mathcal{L}^2$ be the subspace of functions which are square integrable and satisfy a given set of BCs.
The SL problem can then be stated as 
\begin{eqnarray}
    \hat{L} \phi = \lambda \phi
,\end{eqnarray}
and because of this relation we call $L$ the SL operator.
Effectively, we have explicitly rewritten \eref{SL-DE} as an eigenvalue equation.
We now study generally some properties of the SL operator and its spectrum and space of eigenfunctions.

\textbf{Theorem}: \textit{The SL operator is self-adjoint}. Recall that the adjoint $A^{\dagger}$ of an operator $A$ is defined by the equality $(A^{\dagger}f, g) = (f,Ag)$, and a self-adjoint operator is one such that $A^{\dagger} = A$.
The proof is as follows for the SL operator.
Consider the inner product
\begin{align}
    (f,Lg) &= \int_{a}^{b} r(x) f^{*}(x) \hat{L} g(x) \dd{x} = \int_{a}^{b} f^{*}(x) \Bigg[ - \dv{x} p(x) \dv{x} + q(x) \Bigg] g(x) \dd{x} \nonumber \\
    &= -\int_{a}^{b} f^{*}(x) \dv{x} p(x) \dv{x} g(x) \dd{x} + \int_{a}^{b} [ q(x) f(x) ]^{*} g(x) \dd{x} \nonumber \\
    &= -\Big[ f^{*}(x) p(x) g'(x) \Big]_{a}^{b} + \int_{a}^{b} p(x) \dv{f^{*}(x)}{x} \dv{g(x)}{x} \dd{x} + \int_{a}^{b} [ q(x)f(x) ]^{*} g(x) \dd{x} \nonumber \\
    &= \Bigg\{ p(x) \Bigg[ \dv{f^{*}(x)}{x} g(x) - f^{*}(x) \dv{g(x)}{x} \Bigg] \Bigg\}_{a}^{b} + \int_{a}^{b} \Bigg\{ \Bigg[ -\dv{x} p(x) \dv{x} + q(x) \Bigg] f \Bigg\}^{*} g(x) \dd{x} \nonumber \\
    &= \int_{a}^{b} ( \hat{L} f )^{*} g(x) \dd{x} = ( \hat{L} f,g )
.\end{align}
Note that the first term in the second to last line is in fact zero for every set of BCs we mentioned:
\begin{align}
    \Bigg\{ p(x) &\Bigg[ \dv{f^{*}(x)}{x} g(x) - f^{*}(x) \dv{g(x)}{x} \Bigg] \Bigg\}_{a}^{b} \nonumber \\
    &= p(b) \Big[ \dv{f^{*}(b)}{x}g(b) - f^{*}(b) \dv{g(b)}{x} \Big] - p(a) \Big[ \dv{f^{*}(a)}{x}g(a) - f^{*}(a) \dv{g(a)}{x} \Big] \\
    &= 
.\end{align}
For Dirichlet BCs and Neumann BCs, it is straightforward to see this.

https://www.iitg.ac.in/physics/fac/charu/courses/ph402/SturmLiouville.pdf




