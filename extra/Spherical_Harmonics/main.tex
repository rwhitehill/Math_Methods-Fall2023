\chapter{Spherical Harmonics}

This section will be fairly quick since in the previous chapter, we have already laid down most of the foundation.
The spherical harmonics are functions $Y(\theta,\phi)$ which satisfy Laplace's equation, where $\theta$ and $\phi$ are the azimuthal and polar angles, respectively, in spherical coordinates:
\begin{eqnarray}
    r^2 \laplacian Y(\theta,\phi) = \frac{1}{\sin{\theta}} \pdv{\theta} \Big( \sin{\theta} \pdv{Y(\theta,\phi)}{\theta} \Big) + \frac{1}{\sin^2{\theta}} \pdv[2]{Y(\theta,\phi)}{\phi} = 0
.\end{eqnarray}
In the last chapter, we proceeded to solve this equation via a separable ansatz $Y(\theta,\phi) = P(\theta) T(\phi)$ and found that the azimuthal dependence is given by the associated Legendre polynomials $P(\theta) = P_{l}^{m}(\cos{\theta})$ and the polar dependence is given by phase factors $e^{i m \phi}$, where $l \geq 0$ and $-l \leq m \leq l$ are integers.
The full separable solution is then just
\begin{eqnarray}
    Y_{lm}(\theta,\phi) = \mathcal{N}_{lm} P_{l}^{m}(\cos{\theta}) e^{im\phi}
.\end{eqnarray}
The normalization factor is fixed by requiring (as is conventional) that
\begin{eqnarray}
    \int \dd{\Omega} Y_{l'm'}^{*}(\theta,\phi) Y_{lm}^{*}(\theta,\phi) = \delta_{l'l} \delta_{m'm}
.\end{eqnarray}
The orthogonality is already a given with the associated Legendre polynomials and phase factors.
We can determine the normalization factor quite easily then:
\begin{align}
    1 = \int_{0}^{2\pi} \dd{\phi} \int_{0}^{\pi} \dd{\theta} \sin{\theta} |Y_{lm}(\theta,\phi)|^2 &= \mathcal{N}_{lm}^2 (2 \pi) \int_{0}^{\pi} \dd{\theta} \sin{\theta} [P_{l}^{m}(\cos{\theta})]^2 \nonumber \\
                                                                                              &= \mathcal{N}_{lm}^2 (2 \pi) \int_{-1}^{1} \dd{x} [ P_{l}^{m}(x) ]^2 \nonumber \\
                                                                                              &= \mathcal{N}_{lm} \frac{4 \pi}{2l+1} \frac{(l+m)!}{(l-m)!}
.\end{align}
Solving for $\mathcal{N}_{lm}$ then gives the normalized spherical harmonics as
\begin{eqnarray}
    Y_{lm}(\theta,\phi) = \sqrt{\frac{2l+1}{4 \pi} \frac{(l-m)!}{(l+1)!}} P_{l}^{m}(\cos{\theta}) e^{im\phi}
,\end{eqnarray}
and that is the spherical harmonics in a nutshell.

\section{Expansions in spherical harmonics}

One final note is about the completeness of the spherical harmonics.
Consistent with what we claimed in the Sturm-Liouville section, for any square-integrable function $f(\theta,\phi)$, we can expand it in spherical harmonics as
\begin{eqnarray}
    f(\theta,\phi) = \sum_{l=0}^{\infty} \sum_{m=-l}^{l} f_{lm} Y_{lm}(\theta,\phi)
,\end{eqnarray}
where the coefficients are given through the orthonormality condition
\begin{eqnarray}
    f_{lm} = \int \dd{\Omega} Y_{lm}^{*}(\theta,\phi) f(\theta,\phi)
.\end{eqnarray}

If our function is rotationally symmetric about the $z$-axis then $f(\theta,\phi) = f(\theta)$ and only the $m = 0$ term survives, yielding an expansion in Legendre polynomials:
\begin{eqnarray}
    f(\theta) = \sum_{l=0}^{\infty} f_{l} P_{l}(\cos{\theta})
\end{eqnarray}
with some constants absorbed into the coefficients
\begin{eqnarray}
    f_{l} = \frac{2l+1}{2} \int_{0}^{\pi} \dd{\theta} \sin{\theta} P_{l}(\cos{\theta}) f(\theta)
.\end{eqnarray}


\section{Parity}

In the section on associated Legendre polynomials, we introduced the parity operator, which is effectively a spatial inversion operator, and derived the action of this transformation on the associated Legendre functions.
Here, we would like to do the same for the spherical harmonics.
First, we must understand how the parity operator acts on the spherical coordinates themselves.
Under parity, the cartesian coordinates transform trivially with a minus sign.
We then easily obtain
\begin{align}
    r = \sqrt{x^2 + y^2 + z^2} &\rightarrow r \\
.\end{align}
The angles are perhaps a bit trickier since we must consider the domains of the trigonometric functions which define them.
For example,
\begin{eqnarray}
    \phi = \arctan(\frac{y}{x}) \rightarrow \phi' = \arctan(\frac{-y}{-x})
.\end{eqnarray}
It is important not to simply cancel the minus signs without considering the fact that $\arctan{x}$ is only defined on $[-\pi/2,\pi/2]$, so it is only sensitive to the relative sign of $x$ and $y$.
On the other hand, though, the parity operator is a reflection through the origin, taking $(x,y)$ into the point $(-x,-y)$, which is antiparallel to the original position vector projected into the $xy$-plane.
Hence, this reflection through the origin is also equivalent to a $180^{\circ}$ rotation about the $z$-axis, meaning the parity operator takes $\phi \rightarrow \phi + \pi$.
Similarly, for the azimuthal angle
\begin{eqnarray}
    \theta = \arctan(\frac{\sqrt{x^2 + y^2}}{z}) \rightarrow \theta' = \arctan(\frac{\sqrt{x^2 + y^2}}{-z})
.\end{eqnarray}
We can see geometrically that the the parity operator is a reflection about the $xy$-plane with respect to $z$.
Hence, if we define the angle $\vartheta$ as the angle that the position vector makes with the negative $z$-axis, then we have the simple relation $\vartheta + \theta = \pi$, and after the parity operation $\vartheta = \theta$, meaning that $\theta' = \pi - \vartheta = \pi - \theta$.
As a summary, we then have
\begin{eqnarray}
    r \rightarrow r,~ \phi \rightarrow \phi + \pi,~{\rm and}~\theta \rightarrow \pi - \theta
.\end{eqnarray}

Recall the spherical harmonics are given as
\begin{eqnarray}
    Y_{lm}(\theta,\phi) = \sqrt{\frac{2l+1}{4\pi} \frac{(l-m)!}{(l+m)!}} P_{l}^{m}(\cos{\theta}) e^{im\phi}
,\end{eqnarray}
so under parity
\begin{align}
    Y_{lm}(\theta,\phi) \rightarrow Y_{lm}(\pi-\theta,\phi+\pi) = \sqrt{\frac{2l+1}{4\pi}\frac{(l-m)!}{(l+m)!}} P_{l}^{m}(\cos(\pi - \theta)) e^{im(\phi + \pi)}
.\end{align}
Recall that $\cos(\pi - \theta) = -\cos{\theta}$ and $e^{im\pi} = \cos{m\pi} = (-1)^{m}$, so
\begin{align}
    Y_{lm}(\theta,\phi) \rightarrow Y_{lm}(\pi-\theta,\phi+\pi) &= (-1)^{m} \sqrt{\frac{2l+1}{4\pi} \frac{(l-m)!}{(l+m)!}} P_{l}^{m}(-\cos{\theta}) e^{im\phi} \nonumber \\
                                &= (-1)^{l} Y_{lm}(\theta,\phi)
,\end{align}
where we have used $P_{l}^{m}(-x) = (-1)^{l+m} P_{l}^{m}(x)$.







