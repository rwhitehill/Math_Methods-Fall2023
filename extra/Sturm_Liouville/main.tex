\chapter{Sturm-Liouville Differential Equation}

\section{Motivation}

Many of the problems we have studied follow from general properties of the Sturm-Liouville (SL) problem.
In these problems, a system is governed by the differential equation of the form 
\begin{eqnarray}
    \label{eq:SL-DE}
    \dv{x}\Bigg[ p(x) \dv{x} \Bigg] \phi(x) + q(x) \phi(x) =  -\lambda r(x) \phi(x)
\end{eqnarray}
for $x \in [a,b]$, where $a,b \in \R$ and some real functions $p,q,r$, where $p(x),r(x) > 0$ for $x \in (a,b)$\footnote{Note that the endpoints are excluded here!}.
This is in essence an eigenvalue problem, where we would like to determine the eigenvalues $\lambda$ and corresponding eigenfunctions $\phi(x)$ which satisfy \eref{SL-DE} for a given set of $p,q,r$.

For example, the Schr\"{o}dinger equation, Bessel's equation, and Legendre's equation all fall under this category.
Clearly, the particular properties that determines a system's unique behavior depend on $p,q,r$, but the general methods by which we uncover these all follow the same generic trend which follow from general behaviors of systems with a Sturm-Liouville description.


\section{Boundary Conditions}

Since the \eref{SL-DE} is a second-order differential equation, we have two linearly independent solutions $y_{1,2}$ for a given set of functions $p,q,r$, and a general solution $y = \alpha_1 y_1 + \alpha_2 y_2$, where $\alpha_{1,2}$ are constants.
These constants are determined by boundary conditions at $x=a,b$ by specifying (1) $y(a)$ and $y(b)$ [Dirichlet BCs], (2) $y'(a)$ and $y'(b)$ [Neumann BCs], or (3) $c_a y(a) + d_{a} y'(a)$ and $c_{b} y(b) + d_{b}y'(b)$ [Robin BCs].
For the most part, we focus on Dirichlet BCs (e.g. specifying the value of the wave function) and Neumann (e.g. specifying the surface charge density).
It is rare in the textbook problems that a linear combination of the function and its derivative are specified since these linear combinations are not usually related to a physical quantity.
For the development of general properties, though, we will reference these kind of BCs, but Dirichlet and Neumann BCs are recovered by setting $d_{a,b} = 0$ and $c_{a,b} = 0$, respectively.

Finally, in some cases, a fourth distinct set of BCs can be specified, which are called periodic BCs.
As the name suggests we either have $y(a) = y(b)$ or $y'(a) = y'(b)$.

\section{Definitions}

Let us define a couple of terms that will be used to identify the type of equation we consider.

\begin{enumerate}
    \item A \textit{regular} SL system is one such that homogeneous mixed BCs are given: $c_{a}y(a) + d_{a}y'(a) = 0$ and $c_{b}y(b) + d_{b}y'(b) = 0$
    \item A \textit{periodic} SL system is one such that periodic BCs are specified and $p(a) = p(b)$
    \item A \textit{singular} SL system is one where any of the following occur:
    \begin{itemize}
        \item $p(a) = 0$, no BC at $a$ is given, and the BC at $b$ is homogeneous mixed (Note: solutions must be bounded at $x=a$)\footnote{A function $f$ is bounded at $x$ if $|f(x)| < M$ for some $M$.}
        \item $p(b) = 0$, no BC at $b$ is given, and the BC at $a$ is homogeneous mixed (Note: solutions must be bounded at $x = a$)
        \item $p(a) = p(b) = 0$ and no BCs are given (solutions must be bounded at both $x = a,b$.)
        \item $a \rightarrow -\infty$ and $b \rightarrow \infty$ such that the equation is defined on $\R$ (Note: solutions must be square-integrable on $\R$)\footnote{A function $f$ is square-integrable if $\int_{-\infty}^{\infty} |f(x)|^2 < \infty$ (i.e. the integral is convergent and bounded).}
    \end{itemize}
\end{enumerate}


\section{Properties of the Sturm-Liouville System}

\subsection{Sturm-Lioville Operator}

Let $\mathcal{L}^2([a,b],r(x),\dd{x})$ be the Hilbert space of square integrable functions on the interval $[a,b]$ with inner product
\begin{eqnarray}
    (f,g) = \int_{a}^{b} r(x) f^{*}(x) g(x) \dd{x}
.\end{eqnarray}
It is for this reason that $r(x)$ is sometimes denoted a \textit{weight} function.

Denote the linear differential operator
\begin{eqnarray}
    \label{eq:SL-lin-op}
    \hat{L} = -\frac{1}{r(x)} \Bigg[ \dv{x} p(x) \dv{x} + q(x) \Bigg]
,\end{eqnarray}
and let $\mathcal{H} \subset \mathcal{L}^2$ be the subspace of functions which are square integrable and satisfy a given set of BCs.
The SL problem can then be stated as 
\begin{eqnarray}
    \hat{L} \phi(x) = \lambda \phi(x)
,\end{eqnarray}
and because of this relation we call $L$ the SL operator.
Effectively, we have explicitly rewritten \eref{SL-DE} as an eigenvalue equation.
We now study generally some properties of the SL operator and its spectrum and space of eigenfunctions.

\subsection{Facts about solutions and eigenvalues of the Sturm-Lioville problem}
\textbf{Theorem 1}: \textit{The SL operator is self-adjoint}. Recall that the adjoint $A^{\dagger}$ of an operator $A$ is defined by the equality $(A^{\dagger}f, g) = (f,Ag)$, and a self-adjoint operator is one such that $A^{\dagger} = A$.
The proof is as follows for the SL operator.
Consider the inner product
\begin{align}
    (f,Lg) &= \int_{a}^{b} r f^{*} \hat{L} g \dd{x} = \int_{a}^{b} f^{*} \Bigg[ - \dv{x} \Big( p \dv{x} \Big) + q \Bigg] g \dd{x} \nonumber \\
    &= -\int_{a}^{b} f^{*} \dv{x} \Big( p \dv{x} \Big) g \dd{x} + \int_{a}^{b} [ q f ]^{*} g \dd{x} \nonumber \\
    &= -\Big[ f^{*} p g' \Big]_{a}^{b} + \int_{a}^{b} p \dv{f^{*}}{x} \dv{g}{x} \dd{x} + \int_{a}^{b} [ q f ]^{*} g \dd{x} \nonumber \\
    &= \Bigg[ p \Bigg( \dv{f^{*}}{x} g - f^{*} \dv{g}{x} \Bigg) \Bigg]_{a}^{b} + \int_{a}^{b} \Bigg\{ \Bigg[ -\dv{x} p \dv{x} + q \Bigg] f \Bigg\}^{*} g \dd{x} \nonumber \\
    &= \int_{a}^{b} ( \hat{L} f )^{*} g \dd{x} = ( \hat{L} f,g )
.\end{align}
We get rid of the boundary terms in order to have this proof work.
It is a defining property of an SL system that the boundary term (which serves as the BC for the problem) vanishes.
This is straightforward to see for homogeneous Dirichlet BCs ($f(a) = f(b) = g(a) = g(b) = 0$) and Neumann BCs ($f'(a) = f'(b) = g'(a) = g'(b) = 0$) and can be proven for each of the types of SL problems and BCs enumerated above.

\textbf{Theorem 2}: \textit{The eigenvalues of $\hat{L}$ are real}.
Suppose $\phi_{\lambda} \not\equiv 0$ is the function with corresponding eigenvalue $\lambda$.
That is $\hat{L} \phi_{\lambda} = \lambda \phi_{\lambda}$.
Since $\hat{L}$ is self-adjoint, we can write
\begin{align}
    (\hat{L}\phi_{\lambda},\phi_{\lambda}) &= (\phi_{\lambda},\hat{L}\phi_{\lambda}) \nonumber \\
    (\lambda \phi_{\lambda},\phi_{\lambda}) &= (\phi_{\lambda},\lambda \phi_{\lambda}) \nonumber \\
    \lambda^{*} (\phi_{\lambda},\phi_{\lambda}) &= \lambda (\phi_{\lambda},\phi_{\lambda}) \nonumber \\
    \lambda^{*} &= \lambda
\end{align}

\textbf{Theorem 3}: \textit{If $\phi_{\lambda}$ and $\phi_{\mu}$ correspond to distinct eigenvalues $\lambda$ and $\mu$, then $\phi_{\lambda}$ and $\phi_{\mu}$ are orthogonal (i.e. $(\phi_{\lambda},\phi_{\mu}) = 0$)}.
The proof of this fact is straightforward.
The eigenfunctions satisfy $\hat{L} \phi_{\lambda} = \lambda \phi_{\lambda}$ and $\hat{L} \phi_{\mu} = \mu \phi_{\mu}$.
It follows then that
\begin{align}
    (\hat{L}\phi_{\lambda},\phi_{\mu}) &= (\phi_{\lambda},\hat{L}\phi_{\mu}) \nonumber \\
    \lambda(\phi_{\lambda},\phi_{\mu}) &= \mu(\phi_{\lambda},\phi_{\mu})
.\end{align}
Rewriting we have
\begin{eqnarray}
    [\lambda - \mu](\phi_{\lambda},\phi_{\mu}) = 0 \Rightarrow (\phi_{\lambda},\phi_{\mu}) = 0
\end{eqnarray}
since by assumption $\lambda \ne \mu$.

\textbf{Theorem 4}: \textit{The spectrum of $\hat{L}$ is non-degenerate}.
That is to say that if $\phi_1$ and $\phi_2$ correspond to the same eigenvalue, then $\phi_2 = c \phi_2$, or $\phi_1$ and $\phi_2$ are linearly dependent.
We prove this by contradiction.
Suppose that there exists $\phi_1 \ne \phi_2$ such that $\hat{L} \phi_{1,2} = \lambda \phi_{1,2}$.
We then have
\begin{align}
    \phi_2 \hat{L} \phi_1 - \phi_1 \hat{L} \phi_2 &=  -\frac{1}{r(x)} \Big[ \phi_2 \dv{x} \Big( p \phi_1 \Big) - \phi_1' \dv{x} \Big( p \phi_2' \Big) \Big] = 0 \nonumber \\
                                                  &= -\frac{1}{r(x)} \dv{x} \underbrace{ p(x) \Big[ \phi_1' \phi_2 - \phi_1 \phi_2' \Big] }_{W[\phi_1,\phi_2]} = 0
.\end{align}
We then have 
\begin{eqnarray}
    p(x) W[\phi_1(x),\phi_2(x)] = c
.\end{eqnarray}
Notice that for homogeneous BCs,
\begin{eqnarray}
    W[\phi_1(x),\phi_2(x)] = \dv{\phi_1}{x} \phi_2 - \phi_1 \dv{\phi_2}{x} = 0
.\end{eqnarray}
This is simple to see for pure Dirichlet and Neumann BCs since either the function or the derivative is zero at the boundaries.
Thus, we have a separable $1^{\rm st}$ order equation with solution
\begin{eqnarray}
    \phi_1(x) = c\phi_2(x)
.\end{eqnarray}

\textbf{Theorem 5}: \textit{The set of eigenfunctions is a basis for $\mathcal{H}$}.
Equivalently, the set of eigenfunctions $\{ \phi_{\lambda} \} $ is complete.
Let us assume for now that $\hat{L}$ has a countable spectrum, allowing us to label the eigenfunctions by natural numbers such that eigenvalue $\lambda_{n}$ corresponds to eigenfunction $\phi_{n}$.
A rigorous statement of completeness is this: if $\psi(x)$ is any function in $\mathcal{H}$,
\begin{eqnarray}
    \lim_{n \rightarrow N} || \psi(x) - \sum_{k=1}^{n} c_{k} \phi_{k} ||
,\end{eqnarray}
where $N$ is the number of discrete eigenvalues in the spectrum of $\hat{L}$ (possibly infinite).
Note that the coefficients $c_{k} = (\phi_{k},\psi)$ and the norm $|| \cdot ||$ is defined as $|| \psi || = \sqrt{(\psi,\psi)}$.
Another way of stating completeness is that
\begin{eqnarray}
    \sum_{n} \phi_{n}^{*}(x') \phi_{n}(x) = \delta(x - x')
.\end{eqnarray}
This is equivalent since any function $\psi \in \mathcal{H}$ can be expressed as
\begin{eqnarray}
\psi(x) = \int_{a}^{b} \dd{x'} \psi(x') \delta(x - x') = \sum_{n} \phi_{n}(x) \underbrace{ \int_{a}^{b} \dd{x'} \phi_{n}^{*}(x') \psi(x') }_{c_{n}}
.\end{eqnarray}

Note that there was no ``proof'' here.
We really just posited that the eigenfunctions of $\hat{L}$ forms a complete basis.
The proof is quite involved and requires a more formal and advanced treatment than is within the scope of this discussion.
We will assume that the mathematicians who have proven this result are quite competent and will simply take it as fact\footnote{This argument depends fundamentally on the law of large numbers, which is that the probability of a mistake being missed in the proof of this theorem goes to zero as the number of people who validated it goes to infinity. While it may not be satisfying, we at least can rest well knowing that many very smart people have taken a crack at proving this theorem and found it holds, so the chances of this fact being incorrect is effectively zero for our purposes.}.
It is essential that this theorem is true, though, since many of our problems hinge on the completeness of the eigenfunctions and our ability to expand a solution of an arbitrary SL problem in this basis.


\section{Rodrigues' Formula}

In this section, we are looking ahead a bit.
In our studies of the Legendre polynomials, we were told that the $l^{\rm th}$ polynomial can be written as
\begin{eqnarray}
    P_{l}(x) = \frac{1}{2^{l} l!} \dv[l]{x} (x^2 - 1)^{l}
.\end{eqnarray}
At the time, this formula seemed as if it were handed down from the heavens, requiring some divine inspiration that only applies to the case of Legendre polynomials.
In some sense, this is true, but actually, for many physical systems, the eigenfunctions are some kind of polynomials, which in turn have their own ``Rodrigues' formula''.

Let us consider a simplified SL problem of the form
\begin{eqnarray}
    \dv{x} \Big[ p(x) \dv{\phi}{x} \Big] = -\lambda r(x) \phi(x)
.\end{eqnarray}
All we have done here is set $q(x) = 0$.
Let us expand out the derivative in the first term as
\begin{eqnarray}
    p(x) \phi'' + p'(x) \phi' + \lambda r(x) \phi = 0
.\end{eqnarray}
Let us now present $p(x) = r(x) g(x)$.
Our differential equation then becomes
\begin{eqnarray}
    r(x) g(x) \phi'' + [r'(x) g(x) + r(x) g'(x)] \phi' + \lambda r(x) \phi = 0
.\end{eqnarray}
Dividing by $r(x)$,
\begin{eqnarray}
    g(x) \phi'' + \Big[ \frac{r'(x)}{r(x)} g(x) + g'(x) \Big] \phi' + \lambda \phi = 0
.\end{eqnarray}
We have arrived at an alternate (but equivalent) form for the SL equation.
If we have a differential equation of the form
\begin{eqnarray}
    \label{eq:general-2nd-ode}
    g(x) \phi'' + h(x) \phi' + \lambda \phi = 0
,\end{eqnarray}
we can multiply by a weight function $r(x)$ such that
\begin{eqnarray}
    h(x) = \frac{r'(x)}{r(x)} g(x) + g'(x)
,\end{eqnarray}
Solving for the form of $r$ needed to put the \eref{general-2nd-ode} into SL form, we have
\begin{gather}
    \frac{r'(x)}{r(x)} = \frac{1}{r} \dv{r}{x} = \frac{h}{g} - \frac{1}{g}\dv{g}{x} \nonumber \\
    \int \frac{1}{r}\dv{r}{x} \dd{x} = \ln{r(x)} = \int \frac{h(x)}{g(x)} \dd{x} - \ln{g(x)} \nonumber \\
    r(x) = \frac{1}{g(x)} \exp\Bigg( \int \frac{h(x)}{g(x)} \dd{x} \Bigg)
.\end{gather}

At this point, we can now formulate the Rodrigues formula given an SL problem posed as \eref{general-2nd-ode}.
Let us restrict our attention to situations with $g(x) = g_2 x^2 + g_1 x + g_0$ and $h(x) = h_1 x + h_0$ and polynomial solutions to the differential equation of the form
\begin{eqnarray}
    \phi_{n}(x) = \sum_{k=0}^{n} \alpha_{k} x^{k}
.\end{eqnarray}
We will see that these specifications actually encompass many of the different systems we encounter and hence are general enough for our purposes here.
If these restrictions are satisfied, then we claim that we can write
\begin{eqnarray}
    \label{eq:rodrigues-form-gen}
    \phi_{n}(x) = \frac{1}{r(x)} \dv[n]{x} [r(x)g^{n}(x)]
,\end{eqnarray}
which is in fact Rodrigues' formula in full generality.

We now want to embark on a proof of this fact, which ultimately means that we must show that \eref{rodrigues-form-gen} satisfies \eref{general-2nd-ode}.
The first step of the proof is to determine the value $\lambda_{n}$:
\begin{align}
    (g_2x^2 + g_1x + g_0) \sum_{k=2}^{n} k(k-1) \alpha_{k} x^{k-2} + (h_1 x + h_0) \sum_{k=1}^{n} k \alpha_{k} x^{k-1} + \lambda_{n} \sum_{k=0}^{n} \alpha_{k} x^{k} = 0
.\end{align}
Looking at the $n^{\rm th}$ order term, we have
\begin{eqnarray}
    g_2 n(n-1) \alpha_{n} + h_1 n \alpha_{n} + \lambda_{n} \alpha_{n} = 0 ~{\rm or}~ \lambda_{n} = -g_2 n(n-1) - h_1 n
.\end{eqnarray}
Next, observe that
\begin{eqnarray}
    g[r g^{n}]' = g [ r' g^{n} + n r g^{n-1} g' ] = r g^{n} \Big[ \frac{r'}{r} + n \frac{g'}{g} \Big] = r g^{n} [ (n-1)g' + h ]
.\end{eqnarray}
If we differentiate this equation $n + 1$ times and divide by $r$ we have
\begin{align}
    &\frac{1}{r} \dv[n+1]{x} g [ rg^{n} ]' = \frac{1}{r} \dv[n+1]{x} rg^{n} [ (n-1)g' + h ] \nonumber \\
    \frac{1}{r} \sum_{k=0}^{n+1} \binom{n+1}{k} &\dv[k]{g}{x} \dv[n+2-k]{[rg^{n}]}{x} = \frac{1}{r} \sum_{k=0}^{n+1} \binom{n+1}{k} \dv[n+1-k]{[rg^{n}]}{x} \dv[k]{x} [ (n-1)g' + h ] \nonumber \\
    \frac{g}{r} \dv[n+2]{[rg^{n}]}{x} &+ \frac{(n+1)g'}{r} \dv[n+1]{[rg^{n}]}{x} + \frac{n(n+1) g''}{2r} \dv[n]{[rg^{n}]}{x} = \nonumber \\
    &\frac{(n-1)g' + h}{r} \dv[n+1]{[rg^{n}]}{x} + \frac{(n+1)[ (n-1)g'' + h' ]}{r} \dv[n]{[rg^{n}]}{x} \nonumber
.\end{align}
Note that $g^{(3)} = 0$ and $h^{(2)} = 0$ because they are quadratic and linear, respectively, which leaves us with only the few terms above upon the application of Leibniz's rule for differentiation of products:
\begin{align}
    \dv[n]{x} f(x) g(x) = \sum_{k=0}^{n} \binom{n}{k} f^{(k)}(x) g^{(n-k)}(x) = \sum_{k=0}^{n} \frac{n!}{k!(n-k)!} f^{(k)}(x) g^{(n-k)}(x)
.\end{align}
We now combine terms with common derivatives 
\begin{align}
    \frac{g}{r} \dv[n+2]{[rg^{n}]}{x} &= \frac{[(n-1)g' + h] - (n+1)g'}{r} \dv[n+1]{[rg^{n}]}{x} \nonumber \\
    &+ \frac{2(n+1)[(n-1)g'' + h'] - n(n+1)g''}{2r} \dv[n]{[rg^{n}]}{x} \nonumber \\
    &= \frac{-2g' + h}{r} \dv[n+1]{[rg^{n}]}{x} + \frac{(n-2)(n+1)g'' + 2(n+1)h'}{2r} \dv[n]{[rg^{n}]}{x}
.\end{align}
Let us move everything over to the left and rewrite some factors in terms of $\phi_{n}$:
\begin{align}
    \frac{g}{r} \dv[2]{x} [r\phi_{n}] + \frac{2g' - h}{r} \dv{x} [r\phi_{n}] - \Big[ \frac{n^2-n-2}{2} g'' + (n+1) h' \Big] \phi_{n} &= 0 \nonumber \\
    \frac{g}{r} [ r\phi_{n}'' + 2r'\phi_{n}' + r'' \phi_{n} ] + \frac{2g' - h}{r} [ r'\phi_{n} + r\phi_{n}' ] - \Big[ \frac{n^2 - n - 2}{2} g'' + (n+1) h' \Big] \phi_{n} &= 0 \nonumber \\
    g \phi_{n}'' + \Big[ 2 g \frac{r'}{r} + (2g'-h) \Big] \phi_{n}' + \Big[ g \frac{r''}{r} + (2g' - h) \frac{r'}{r} - \Big( \frac{n^2 - n - 2}{2} g'' + (n+1)h' \Big) \Big] \phi_{n} &= 0
.\end{align}
Recall that $r'/r = (h - g')/g$ and similarly
\begin{eqnarray}
    g\frac{r''}{r} = -\frac{r'}{r} (2g' - h) - (g'' - h')
.\end{eqnarray}
Plugging these into the equation above:
\begin{gather}
    g \phi_{n}'' + h \phi_{n}' + \Big[ \frac{r'}{r}(h - 2g') - (g'' - h') + \frac{r'}{r}(2g' - h) - \Big( \frac{n^2 - n - 2}{2} g'' + (n+1)h' \Big) \Big] \phi_{n} = 0 \nonumber \\
    g \phi_{n}'' + h\phi_{n}' - \Big( \frac{n^2 - n}{2} g'' + n h' \Big) \phi_{n} = 0
.\end{gather}
Almost there!
Let us just recall that $\lambda_{n} = -[ g'' n(n-1)/2 + n h']$, meaning
\begin{eqnarray}
    g \phi_{n}'' + h \phi_{n}' + \lambda_{n} \phi_{n} = 0
,\end{eqnarray}
which is exactly the simplified SL form we assumed $\phi_{n}$ satisfied in the first place, proving Rodrigues' formula\footnote{Note that there are usually pre-factors in front of the derivative when considering specific cases. Scaling by a constant (i.e. $\phi_{n} \rightarrow c \phi_{n}$) does not change anything except for normalization.}.


\section{Schlaefli integral and generating functions}

We know from complex analysis that the $n^{\rm th}$ derivative of an analytic function is just
\begin{eqnarray}
    f^{(n)}(x) = \frac{n!}{2 \pi i} \oint_{C} \frac{f(z)}{(z - x)^{n+1}} \dd{x}
,\end{eqnarray}
where $z$ is in the region bounded by $C$ and $f$ is analytic on and inside $C$.
It quickly follows then that we can transform the generic Rodrigues formula such that
\begin{eqnarray}
    \phi_{n}(x) = \frac{1}{r(x)} \frac{n!}{2 \pi i} \oint_{C} \frac{r(z) g^{n}(z)}{(z - x)^{n+1}} \dd{z}
.\end{eqnarray}

In some cases, the integral representation of the SL solutions is useful in deriving a generating function $G(x,t)$ such that its power series expansion\footnote{$G(x,t)$ here should not be confused with a Green function of some sort. Typically generating functions are denoted by $g(x,t)$, but alas, we have already used $g$ in a way that would make things confusing by reusing it.}
\begin{eqnarray}
    G(x,t) = \sum_{n} c_{n} \phi_{n}(x) t^{n}
.\end{eqnarray}
That is, the solutions to the SL problem (with some scalar) are just the coefficients of the expansion.
If we put in the Schlaefli integral representation for $\phi_{n}(x)$, we have
\begin{eqnarray}
    G(x,t) = \sum_{n} c_{n} t^{n} \frac{n!}{2 \pi i} \oint_{C} \frac{r(x) g^{n}(x)}{(z - x)^{n+1}} \dd{z}
.\end{eqnarray}
This is not always useful since the integration must be possible to perform analytically, but there are a few cases of interest to us where we can use this to write down closed forms for the generating function.
With such a function then, we will be able to write down recurrence relations that are useful in solving many different types of problems.





