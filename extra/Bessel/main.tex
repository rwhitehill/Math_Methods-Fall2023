\chapter{Bessel's equation and Bessel functions}


We saw in the last few sections how the associated Legendre polynomials arose when solving Laplace's equation in spherical coordinates.
In this section, we will see Bessel functions when solving Laplace's equation in cylindrical coordinates:
\begin{eqnarray}
    \frac{1}{s} \pdv{s} \Big( s \pdv{\psi}{s} \Big) + \frac{1}{s^2} \pdv[2]{\psi}{\phi} + \pdv[2]{\psi}{z} = 0
.\end{eqnarray}
Again, we pose a separable ansatz $\psi(s,\phi,z) = R(s)T(\phi)Z(z)$, and plugging this in, we find
\begin{eqnarray}
    \frac{1}{s R} \dv{s} \Big( s \dv{R}{s} \Big) + \frac{1}{s^2 T} \dv[2]{T}{\phi} + \frac{1}{Z}\dv[2]{Z}{z} = 0
.\end{eqnarray}
It is clear from this that the two terms are independent of $z$, but the sign of this term is not immediately clear and depends on the boundary conditions of our particular problem.
Let us write 
\begin{eqnarray}
    \frac{1}{Z} \dv[2]{Z}{z} = k^2
\end{eqnarray}
with $k$ not necessarily being real, meaning that $k^2$ can be any real number (positive, negative, or zero).
Thus,
\begin{align}
    \frac{s}{R} \dv{s} \Big( s \dv{R}{s} \Big) + k^2 s^2 + \frac{1}{T} \dv[2]{T}{\phi} = 0
.\end{align}
The form of this term is exactly as in the spherical case since we must have $T(\phi + 2\pi) = T(\phi)$, meaning that
\begin{eqnarray}
    \frac{1}{T} \dv[2]{T}{\phi} = -\nu^2
\end{eqnarray}
for $\nu \in \Z$.
The equation for the $s$-dependence then becomes
\begin{eqnarray}
    s \dv{s} \Big( s \dv{R}{s} \Big) + (k^2 s^2 - \nu^2) R = 0
.\end{eqnarray}
This equation is actually of the Sturm-Liouville type, but it will be easier to discuss after we form a solution.
If we make the substitution $x = ks$, then the we have Bessel's equation
\begin{eqnarray}
    x \dv{x}\Big( x \dv{R}{x} \Big) + (x^2 - \nu^2)R = 0
.\end{eqnarray}


\section{Bessel function of the first kind}

As in the section on Legendre polynomials, we write $R$ in a series as
\begin{eqnarray}
    R(x) = \sum_{n=0}^{\infty} a_{n} x^{n + \gamma}
,\end{eqnarray}
with $a_0 \ne 0$, which yields
\begin{gather}
    \sum_{n=0}^{\infty} [ (n + \gamma)^2 - \nu^2] a_{n} x^{n+\gamma} + \sum_{n=0}^{\infty} a_{n} x^{n+\gamma+2} = 0 \nonumber \\
    (\gamma^2 - \nu^2)a_0 + [(\gamma+1)^2 - \nu^2] a_1 x + \sum_{n=0}^{\infty} \Big[ [(n+2+\gamma)^2 - \nu^2]a_{n+2} + a_{n} \Big] x^{n+\gamma+2} = 0
.\end{gather}
In order to have this expression be identically zero, we need
\begin{align}
    \gamma &= \pm \nu \\
    a_1 &= 0 \\
    a_{n+2} &= -\frac{1}{(n+2+\gamma)^2 - \gamma^2}a_{n} \nonumber \\
            &= -\frac{1}{(n+2)(n+2+2\gamma)} a_{n}
.\end{align}
Note that by the recurrence relation all the odd terms are zero, so we can rewrite $n \rightarrow 2(n-1)$ such that
\begin{align}
    a_{2n} &= -\frac{1}{4n(n+\gamma)} a_{2(n-1)} = \Big( - \frac{1}{4} \Big)^{n} \frac{1}{[n(n-1)\ldots(2)(1)][(n+\gamma)(n+\gamma-1)\ldots(\gamma+1)]} a_0 \nonumber \\
           &= (-1)^{n} \Big( \frac{1}{2} \Big)^{2n} \frac{\gamma!}{n!(n+\gamma)!} c_0
,\end{align}
and if we choose $a_0 = 1/[2^{\gamma} \Gamma(\gamma+1)]$ (this is only so that we can factor some terms and avoid carrying an overall factor), we obtain Bessel's function of the first kind:
\begin{align}
    J_{\nu}(x) = \Big( \frac{x}{2} \Big)^{\nu} \sum_{n=\min(0,\nu)}^{\infty} \frac{(-1)^{n}}{n!(n+\nu)!} \Big( \frac{x}{2} \Big)^{2n}
,\end{align}
where we allow $\nu$ to be any integer\footnote{Actually, there are other systems with different BCs and behaviors which do not restrict $\nu$ to the integers, and since the series solution did not depend on $\nu$ other than in the indicial equation, this is true for any $\nu$ and $(n+\nu)! \rightarrow \Gamma(n+\nu+1)$}.

At this point, we cannot quite claim victory in the solution of Laplace's equation.
if $\nu$ is in fact an integer, then $J_{\nu}$ and $J_{-\nu}$ are linearly dependent, meaning that these do not form a complete set of solutions to the Bessel equation.
We can show this as follows.
Let $\nu > 0$.
Then,
\begin{align}
    J_{-\nu}(x) = \sum_{n=-\nu}^{\infty} \frac{(-1)^{n}}{n!(n-\nu)!} \Big( \frac{x}{2} \Big)^{2n-\nu} = (-1)^{\nu} J_{\nu}(x)
.\end{align}

{\color{red} This needs to be refined a bit. Motivate index for $-\nu$ better.}



