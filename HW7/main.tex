\def\duedate{\today}
\def\HWnum{7}
\input{../preamble.tex}

\begin{document}

\prob{1}{

Expand the following function into the Fourier series:
\begin{eqnarray}
    I_{1}(x) = \sum_{n=-\infty}^{\infty} \delta(x - 2 \pi n)
.\end{eqnarray}

}

\sol{

We can use the complex form of the Fourier series
\begin{eqnarray}
    I_1(x) = \sum_{n=-\infty}^{\infty} c_{n} e^{inx}
\end{eqnarray}
on the interval $[-\pi,\pi]$, where
\begin{eqnarray}
    c_{n} = \frac{1}{2\pi} \int_{-\pi}^{\pi} I_1(x) e^{-inx} \dd{x} = \frac{1}{2\pi}
.\end{eqnarray}
Hence, on $[-\pi,\pi]$, 
\begin{eqnarray}
    I_{1}(x) = \sum_{n=-\infty}^{\infty} \frac{1}{2\pi} e^{inx}
.\end{eqnarray}

We can actually extend this result to any interval $[(2m-1)\pi,(2m+1)\pi]$ by noticing that 
\begin{eqnarray}
    \int_{a-\pi}^{a+\pi} e^{i(n-n')x} = 2\pi \delta_{nn'}
.\end{eqnarray}
Thus, on an interval centered at $2m\pi$ of width $2\pi$
\begin{eqnarray}
    c_{n} = \frac{1}{2\pi} \int_{(2m-1)\pi}^{(2m+1)\pi} I_1(x) e^{-inx} = \frac{1}{2\pi} e^{-in(2m\pi)} = \frac{1}{2\pi}
\end{eqnarray}
since $2nm\pi$ is an integer multiple of $2 \pi$.
Therefore,
\begin{eqnarray}
    \eqbox{ I_1(x) = \sum_{n=-\infty}^{\infty} \frac{1}{2\pi} e^{inx} }
\end{eqnarray}
on each interval of width $2\pi$ centered at $2m\pi$.
Since this result is independent of $m$, we can generally apply this result for any $x \in \reals$.


}


\prob{2}{

Expand the following real function into the Fourier series:
\begin{eqnarray}
    I_2(x) = \sum_{n=-\infty}^{\infty} \frac{1}{a^2 + (x - 2 \pi n)^2}
.\end{eqnarray}

}

\sol{

Observe that this function is $2\pi$-periodic:
\begin{eqnarray}
    \begin{aligned}
        f(x + 2\pi) &= \sum_{n=-\infty}^{\infty} \frac{1}{a^2 + [(x+2\pi) - 2\pi n]^2} \sum_{n=-\infty}^{\infty} \frac{1}{a^2 + [x - 2 \pi (n-1)]^2} \\
                    &= \sum_{n=-\infty}^{\infty} \frac{1}{a^2 + (x - 2 \pi n)^2}
    ,\end{aligned}
\end{eqnarray}
where in the last step, we took $n - 1 \rightarrow n$, which leaves the sum unchanged since the index set spans all integers.
Furthermore, notice that this is an even function since
\begin{eqnarray}
    f(-x) = \sum_{n=-\infty}^{\infty} \frac{1}{a^2 + (x + 2 \pi n)^2} = \sum_{n=-\infty}^{\infty} \frac{1}{a^2 + (x - 2 \pi n)^2}
,\end{eqnarray}
where in the last step we took $n \rightarrow -n$, which again leaves the sum unchanged for the same reason as above.

The Fourier transform can then be written 

}


\prob{3}{

Calculate the following function defined by the Fourier series:
\begin{eqnarray}
    I_6(x) = \sum_{n=1}^{\infty} a_{n} \sin{nx}
.\end{eqnarray}
whith $a_{n} = e^{-kn}$ and $k > 0$.

}

\sol{

We can write
\begin{eqnarray}
\begin{aligned}
    \sum_{n=1}^{\infty} \Big( e^{-k + ix} \Big)^{n} &= \frac{1}{1 - e^{-k+ix}} = \frac{1 - e^{-k-ix}}{(1-e^{-k+ix})(1-e^{-k-ix})} \\
                                                    &= \frac{(1 - e^{-k} \cos{x}) + i e^{-k} \sin{x}}{1 - 2e^{-k}\cos{x} + e^{-2k}}
\end{aligned}
.\end{eqnarray}
Observe then that
\begin{eqnarray}
    \begin{aligned}
        I_6(x) &= \sum_{n=1}^{\infty} e^{-kn} \sin{nx} = \sum_{n=0}^{\infty} e^{-kn}\sin{nx} \\
               &= \Im \Big\{ \sum_{n=0}^{\infty} e^{(-k+ix)n} \Big\} = \frac{e^{-k} \sin{x}}{e^{-2k} \sin^2{x} + (1 - e^{-k} \cos{x})^2}
    .\end{aligned}
\end{eqnarray}

}


\prob{4}{

Expand the following function in the Fourier integral
\begin{eqnarray}
    I_1(x) = \frac{A}{x^2 + a^2} + \frac{Bx}{(x^2 + b^2)^2}
.\end{eqnarray}

}

\sol{

We can write
\begin{eqnarray}
    I_1(x) = \frac{1}{2\pi} \int_{-\infty}^{\infty} C(\omega) e^{-i \omega x} \dd{\omega}
,\end{eqnarray}
where
\begin{align}
    C(\omega) &= \int_{-\infty}^{\infty} I_1(x) e^{i \omega x} \dd{x} \\
              &= A \int_{-\infty}^{\infty} \frac{e^{i \omega x}}{x^2 + a^2} \dd{x} + B \int_{-\infty}^{\infty} \frac{x e^{i \omega x}}{(x^2 + b^2)^2} \dd{x}
.\end{align}
Both of these integrals can be solved by analytically continuing the integrands to the complex plane and using contour integration.
Note that we must be careful in choosing the right contour.
Observe that $e^{i \omega z} = e^{i \omega R \cos{\phi}}e^{-R \omega \sin{\phi}}$, where $z = R \cos{\phi} + i R\sin{\phi}$ along the semi-circle enclosing either the upper or lower half-plane.
Hence, if $\omega > 0$, then enclosing in the upper half-plane gives no contribution along the semi-circle at infinity since $R \rightarrow \infty$ and $\sin{\phi} \geq 0$, but if $\omega < 0$, then we must enclose the poles in the lower half-plane so that the real exponential factor goes to zero at $R \rightarrow \infty$.
We can, however, relate the two contributions by observing that $C(\omega) = C^{*}(-\omega)$, meaning that we only have to consider the case $\omega > 0$ explictly and can obtain the case $\omega < 0$ essentially for free.

If $\omega > 0$, then we enclose the poles in the upper half plane and obtain
\begin{align}
    C(\omega) &= A (2 \pi i) \frac{e^{i \omega z}}{z + ia} \Big|_{z = ia} + B (2 \pi i) \Big[ \dv{z} \frac{z e^{i \omega z}}{(z + ib)^2} \Big]_{z = ib} \nonumber \\
              &= A (2 \pi i) \frac{e^{-\omega a}}{2ia} + B (2 \pi i) \frac{[1 + i\omega z](z + ib)^2 - 2 z (z + ib)}{(z+ib)^{4}} e^{i \omega z} \Big|_{z = ib} \nonumber \\
              &= \frac{A}{a} \pi e^{-\omega a} + B(2 \pi i) \frac{[1 + i\omega(ib)](2ib)^2 - 2(ib)(2ib)}{(2ib)^{4}} e^{-\omega b} \nonumber \\
              &= \frac{A}{a} \pi e^{-\omega a} - B (2 \pi i) \frac{\omega b}{(2ib)^2} e^{-\omega b} \nonumber \\
              &= \pi \Big( \frac{A}{|a|} e^{-\omega |a|} + i \frac{B \omega}{2|b|} e^{-\omega |b|} \Big)
.\end{align}
Notice that we have put in the absolute value by hand since the result should be independent of the signs of $a$ and $b$.
Really, it is because $z^2 + a^2 = (z + i|a|)(z - i|a|)$, and the pole in the upper half-plane is $z = i|a|$.

We can also then write
\begin{eqnarray}
\begin{aligned}
    I_1(x) &= \frac{1}{2 \pi} \int_{-\infty}^{\infty} C(\omega) e^{-i \omega x} \dd{\omega} = \frac{1}{2} \int_{0}^{\infty} \big[ C(\omega) e^{-i \omega x} + C(-\omega) e^{i \omega x} \big] \\
           &= \frac{1}{2 \pi} \int_{0}^{\infty} \big[ C(\omega) e^{-i\omega x} + C^{*}(\omega) e^{i\omega x} \big] \dd{\omega} \\
           &= \frac{1}{\pi} \int_{0}^{\infty} \Big[ \Re\{ C(\omega) \} \cos{\omega x} - \Im\{ C(\omega) \} \sin{\omega x} \Big] \dd{\omega} \\
           &= \int_{0}^{\infty} \Bigg[ \frac{A}{|a|} e^{-\omega|a|} \cos{\omega x} - \frac{B}{2|b|} \omega e^{-\omega|b|} \sin{\omega x} \Bigg] \dd{\omega}
.\end{aligned}
\end{eqnarray}



}


\prob{5}{

A particle of mass $m$ and charge $q$ moves in a parabolic potential $k(x^2 + y^2)/2$ in the $xy$ plane.
Find how the frequency $\omega_0 = (k/m)^{1/2}$ of this 2D harmonic oscillator changes if a uniform magnetic field $H$ is applied along the $z$-axis.
The $x$ and $y$ components of the equation of motion 
\begin{eqnarray}
    m \ddot{\va*{r}} - q[\dot{\va*{r}} \cross \va*{H}]/c + k \va*{r} = 0
\end{eqnarray}
yield two coupled differential equations:
\begin{eqnarray}
\begin{aligned}
    \ddot{x} - \omega_{L} \dot{y} + \omega_0^2 x &= 0 \\
    \ddot{y} + \omega_{L} \dot{x} + \omega_0^2 y &= 0
,\end{aligned} 
\end{eqnarray}
where $\omega_{L} = qH/mc$ is the Larmor frequency.

}

\sol{}

\prob{6}{

Solve the following nonuniform differential equation using the Fourier transform
\begin{eqnarray}
    y'' - q^2 y = Ae^{-a|x|}
,\end{eqnarray}
where $q, a > 0$ and $A$ are real parameters.
Find a continuous solution satisfying the boundary conditions $y(\pm \infty) = 0$.

}






\end{document}
