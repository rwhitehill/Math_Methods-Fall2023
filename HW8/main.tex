\def\duedate{\today}
\def\HWnum{8}
\documentclass[10pt,a4paper]{book}

% custom section formatting
\usepackage{titlesec}
\titleformat{\chapter}[display]
{\normalfont\Large\filcenter\sffamily}
{\titlerule[1pt]%
\vspace{1pt}%
\titlerule
\vspace{1pc}%
\LARGE\MakeUppercase{\chaptertitlename} \thechapter}
{1pc}
{\titlerule
\vspace{1pc}%
\Huge}

% appendix handling
\usepackage[toc,page]{appendix}
    
% encoding for file and font
\usepackage[utf8]{inputenc}
\usepackage[T1]{fontenc}

% math formatting/tools
\usepackage{amsmath}
\usepackage{amssymb}
\usepackage{mathtools}
\usepackage[arrowdel]{physics}
\usepackage{dsfont}

\newcommand{\R}{\mathbb{R}}
\newcommand{\Z}{\mathbb{Z}}
\newcommand{\N}{\mathbb{N}}
\newcommand{\Q}{\mathbb{Q}}
\newcommand{\C}{\mathbb{C}}

% unit formatting
\usepackage{siunitx}
\AtBeginDocument{\RenewCommandCopy\qty\SI}

% figure formatting/tools
\usepackage{graphicx}
\usepackage{float}
\usepackage{subcaption}
\usepackage{multirow}
\usepackage{import}
\usepackage{pdfpages}
\usepackage{transparent}
\usepackage{currfile}

\NewDocumentCommand\incfig{O{1} m}{
    \def\svgwidth{#1\textwidth}
    \import{./Figures/\currfiledir}{#2.pdf_tex}
}

\newcommand{\bef}{\begin{figure}[h!tb]\centering}
\newcommand{\eef}{\end{figure}}

\newcommand{\bet}{\begin{table}[h!tb]\centering}
\newcommand{\eet}{\end{table}}

% hyperlink references 
\usepackage{hyperref}
\hypersetup{
    colorlinks=true,
    linkcolor=blue,
    filecolor=magenta,
    urlcolor=cyan,
    pdftitle={Physics 1 Notes},
    pdfauthor={Richard Whitehill},
    pdfpagemode=FullScreen
}
\urlstyle{same}

\newcommand{\eref}[1]{Eq.~(\ref{eq:#1})}
\newcommand{\erefs}[2]{Eqs.~(\ref{eq:#1})--(\ref{eq:#2})}

\newcommand{\fref}[1]{Fig.~(\ref{fig:#1})}
\newcommand{\frefs}[2]{Fig.~(\ref{fig:#1})--(\ref{fig:#2})}

\newcommand{\aref}[1]{Appendix~(\ref{app:#1})}
\newcommand{\sref}[1]{Section~(\ref{sec:#1})}
\newcommand{\srefs}[2]{Sections~(\ref{sec:#1})-(\ref{sec:#2})}

\newcommand{\tref}[1]{Table~(\ref{tab:#1})}
\newcommand{\trefs}[2]{Table~(\ref{tab:#1})--(\ref{tab:#2})}

% tcolorbox formatting/definitions
\usepackage[most]{tcolorbox}
\usepackage{xcolor}
\usepackage{xifthen}
\usepackage{parskip}

\definecolor{peach}{rgb}{1.0,0.8,0.64}

\DeclareTColorBox[auto counter, number within=chapter]{defbox}{O{}}{
    enhanced,
    boxrule=0pt,
    frame hidden,
    borderline west={4pt}{0pt}{green!50!black},
    colback=green!5,
    before upper=\textbf{Definition \thetcbcounter \ifthenelse{\isempty{#1}}{}{: #1} \\ },
    sharp corners
}

\newcommand*{\eqbox}{\tcboxmath[
    enhanced,
    colback=black!10!white,
    colframe=black,
    sharp corners,
    size=fbox,
    boxsep=8pt,
    boxrule=1pt
]}

\newtcolorbox[auto counter, number within=chapter]{exbox}{
    parbox=false,
    breakable,
    enhanced,
    sharp corners,
    boxrule=1pt,
    colback=white,
    colframe=black,
    before upper= \textbf{Example \thetcbcounter:}\,,
    before lower= \textbf{Solution:}\,,
    segmentation hidden
}

\newtcolorbox{resbox}{
    enhanced,
    colback=black!10!white,
    colframe=black,
    boxrule=1pt,
    boxsep=0pt,
    top=2pt,
    ams nodisplayskip,
    sharp corners
}


\begin{document}

\prob{1}{

Find the Laplace transform of the following function:
\begin{eqnarray}
    I(t) = t^{n} e^{-at}, \quad a > 0 ~{\rm and~even}~n
.\end{eqnarray}

}

\sol{

The Laplace transform of a function $f(t)$ is generally given as
\begin{eqnarray}
    \mathcal{L}\{ f(t) \} = \int_{0}^{\infty} f(t) e^{-st} \dd{t}
,\end{eqnarray}
where the values of $s$ are such that the integral is convergent.
Thus,
\begin{eqnarray}
\begin{aligned}
    \mathcal{L}\{ I(t) \} &= \int_{0}^{\infty} t^{n} e^{-(a+s)t} \dd{t} = (-1)^{n} \dv[n]{(a+s)} \int_{0}^{\infty} e^{-(a+s)t} \dd{t} \\
                          &= (-1)^{n} \dv[n]{(a+s)} \frac{1}{a+s} = \frac{n!}{(a+s)^{n+1}}
.\end{aligned}
\end{eqnarray}
This matches the result we have in our library.
That is, $\mathcal{L}\{ t^{n} \} = n!/s^{n+1}$ and for a general function $f(t)$, $\mathcal{L}\{ f(t)e^{-at} \} = F(s + a)$.
Collecting these results, we end up with the same solution, which is that
\begin{eqnarray}
    \eqbox{ \mathcal{L}\{ t^{n} e^{-at} \} = \frac{n!}{(s + a)^{n+1}} }
.\end{eqnarray}

}


\prob{2}{

Solve the following equation by the Laplace transform
\begin{eqnarray}
    \ddot{y} + 2\lambda \dot{y} + \omega_0^2 y = 0
,\end{eqnarray}
where $y(0) = 0$ and $\dot{y}(0) = v$.

}

\sol{

If we take the Laplace transform of the equation\footnote{Strictly speaking, this means that if we have a differential equation $D f = g$, where $D$ is a linear differential operator, then its ``Laplace transform'' is $\mathcal{L}\{ D f \} = G(s)$.}, we find
\begin{eqnarray}
    \mathcal{L}\{ \ddot{y} + 2 \lambda \dot{y} + \omega_0^2 y \} = [ s^2 Y(s) - v ] + 2 \lambda [ sY(s) ] + \omega_0^2 Y(s) = 0
.\end{eqnarray}
Solving for $Y(s)$ gives
\begin{eqnarray}
    Y(s) = \frac{v}{s^2 + 2\lambda s + \omega_0^2} = \frac{v}{(s + \lambda)^2 + (\omega_0^2 - \lambda^2)}
.\end{eqnarray}
From our ``library'' we have
\begin{align}
    \mathcal{L}\{ e^{-at}\cos{bt} \} &= \frac{s+a}{(s+a)^2 + b^2} \\
    \mathcal{L}\{ e^{-at}\sin{bt} \} &= \frac{b}{(s+a)^2 + b^2}
.\end{align}
Thus, if we let $a = \lambda$ and $b = \sqrt{\omega_0^2 - \lambda^2}$, then
\begin{eqnarray}
    \eqbox{ y(t) = \frac{v}{\sqrt{\omega_0^2 - \lambda^2}} e^{-\lambda t} \sin(\sqrt{\omega_0^2 - \lambda^2} \, t) }
.\end{eqnarray}
This is the solution in general for any $\omega_0$ and $\lambda$ (subject to the initial conditions above), but we can specify the relative values of $\omega_0$ and $\lambda$ in order to rewrite the solution in terms of explicitly real factors .
First, if $\omega_0^2 > \lambda^2$.
In this case, the solution does not change since all the constants are already real.

Second, if $\omega_0^2 = \lambda^2$, we have an indeterminate form, which is defined as the limit $\sqrt{\omega_0^2 - \lambda^2} \rightarrow 0$.
Clearly, this is of the form $\lim_{x \rightarrow 0} \sin{ax}/x = a$, leaving us with
\begin{eqnarray}
    \eqbox{ y(t) = v t e^{-\lambda t} }
.\end{eqnarray}

Finally, if $\omega_0^2 < \lambda^2$, we have $\sqrt{\omega_0^2 - \lambda^2} = i \sqrt{\lambda^2 - \omega_0^2}$, and using $\sin{ix} = i\sinh{x}$, we have
\begin{eqnarray}
    \eqbox{ y(t) = \frac{v}{\sqrt{\lambda^2 - \omega_0^2}} e^{-\lambda t}\sinh( \sqrt{\lambda^2 - \omega_0^2} \, t ) }
.\end{eqnarray}

Now that we have enumerated all the possible combinations, we are done, and we see that we have recovered all the forms of the solution to a generic homogeneous linear, second-order ordinary differential equation with constant coefficients with initial conditions $y(0) = 0$ and $\dot{y}(0) = v$.


}


\prob{3}{

A unit vector $\vu*{n}$ makes angles $\theta$ and $\alpha$ with the Cartesian axes $z$ and $x$, respectively, and a unit vector $\vu*{n}'$ makes angles $\theta'$ and $\alpha'$ with $z$ and $x$, respectively.
Find $\cos{\varphi}$, where $\varphi$ is the angle between $\vu*{n}$ and $\vu*{n}'$.

}

\sol{

    Note that if a vector $\va*{A}$ ``makes an angle'' $\phi$ with some axis defined by a unit vector $\vu*{e}$, this means that $\va*{A} \cdot \va*{e} = |\va*{A}| \cos{\phi}$, and therefore the component of $\va*{A}$ along $\vu*{e}$ is just $A_{e} = |\va*{A}| \cos{\phi}$.

Using this fact, we can write
\begin{align}
    \vu*{n} &= \cos{\alpha} \vu*{x} + \sqrt{1 - \cos^2{\alpha} - \cos^2{\theta}} \, \vu*{y} - \cos{\theta} \vu*{z} \\
    \vu*{n}' &= \cos{\alpha'} \vu*{x} + \sqrt{1 - \cos^2{\alpha'} - \cos^2{\theta'}} \, \vu*{y} + \cos{\theta'} \vu*{z}
.\end{align}
Through the dot product, we have
\begin{eqnarray}
    \vu*{n} \cdot \vu*{n}' = \cos{\varphi} = \cos{\alpha}\cos{\alpha'} + \sqrt{(1 - \cos^2{\alpha} - \cos^2{\theta})(1 - \cos^2{\alpha'} - \cos^2{\theta'})} + \cos{\theta}\cos{\theta'}
.\end{eqnarray}
Let us simplify the term under the radical:
\begin{eqnarray}
\begin{aligned}
    (1 &- \cos^2{\alpha} - \cos^2{\theta})(1 - \cos^2{\alpha'} - \cos^2{\theta'}) =
\end{aligned}
.\end{eqnarray}

}


\prob{4}{

Find a scalar function $\varphi(r)$ of $r = |\va*{r}|$ which satisfies the equation
\begin{eqnarray}
    \div{[ \varphi(r) \va*{r} ]} = 0
.\end{eqnarray}

}

\sol{

We can write the equation above as
\begin{eqnarray}
    \label{eq:simplified-4}
    \div{ [ \varphi(r) \va*{r} ] } = \varphi(r) \div{\va*{r}} + \grad{\varphi(r)} \cdot \va*{r} = 3 \varphi(r) + x \pdv{\varphi(r)}{x} + y \pdv{\varphi(r)}{y} + z \pdv{\varphi(r)}{z} = 0
.\end{eqnarray}

Recall the transformation between the Cartesian $(x,y,z)$ and spherical $(r,\phi,\theta)$ bases in $\reals^{3}$ is just
\begin{eqnarray}
    \begin{cases}
        x = r \sin{\theta} \cos{\phi} \\
        y = r \sin{\theta} \sin{\phi} \\
        z = r \cos{\theta}
    \end{cases}
    \Leftrightarrow
    \begin{cases}
        r = \sqrt{x^2 + y^2 + z^2} \\
        \phi = \arctan(y/x) \\
        \theta = \arctan(z/\sqrt{x^2 + y^2})
    .\end{cases}
\end{eqnarray}
We can use this to transform the last three terms of \eref{simplified-4} to an explicit representation in terms of spherical components.
It suffices to do this for the $x$-component and generalize to the other two:
\begin{eqnarray}
    \pdv{\varphi(r)}{x} = \pdv{r}{x} \pdv{\varphi}{r} = \frac{x}{r} \dv{\varphi}{r}
,\end{eqnarray}
Plugging this into \eref{simplified-4} then gives us
\begin{eqnarray}
    \div{ [ \varphi(r) \va*{r} ] } = 3 \varphi(r) + \frac{x^2 + y^2 + z^2}{r} \dv{\varphi(r)}{r} = 3 \varphi(r) + r \dv{\varphi(r)}{r} = 0
.\end{eqnarray}
This is just a separable differential equation, which gives
\begin{eqnarray}
    \eqbox{ \varphi(r) = \frac{A}{r^3} }
,\end{eqnarray}
where $A$ is a constant determined by some boundary condition.

}


\prob{5}{

    Calculate the following: (1) $\div{[(\va*{a} \cdot \va*{r}) \va*{b}]}$, (2) $\curl{[(\va*{a} \cdot \va*{r})\va*{b}]}$, (3) $\div{[\va*{a} \cross \va*{r}]}$, (4) $\curl{(\va*{a} \cross \va*{r})}$, (5) $\div{[\va*{r} \cross (\va*{a} \cross \va*{r})]}$, where $\va*{a}$ and $\va*{b}$ are constant vectors.

}

\sol{

There are a few ways of approaching these kinds of problems.
One way is to do the calculus and algebra via brute-force in its full glory.
Another utilizes a nicer and more compact notation through tensors and the Einstein summation notation.
The summation notation is that we can write sums like $\va*{A} = \sum_{i=1}^{3} A_{i} \vu*{e}_{i} = A_{i} \vu*{e}_{i}$, dropping the explicit sum and implicitly understanding that repeated indices are summed over.
Note that $i=1,2,3$ corresponds to the $x,y,z$ components of $\va*{A}$.
Additionally, we can define the Kronecker-delta symbol (effectively the three-dimensional analogue to the metric tensor) as
\begin{eqnarray}
    \delta_{ij} = \begin{cases}
        1 & {\rm if} ~ i = j \\
        0 & {\rm if} ~ i \ne j
    .\end{cases}
\end{eqnarray}
This is useful for defining the dot product between vectors as
\begin{eqnarray}
    \va*{A} \cdot \va*{B} = A_{i}B_{i} = A_{i}B_{j} \delta_{ij}
.\end{eqnarray}
We also introduce the Levi-civita symbol as
\begin{eqnarray}
    \epsilon_{ijk} = \begin{cases}
        1 & {\rm if} ~ ijk \in \{ (123), (231), (312) \} \\
        1 & {\rm if} ~ ijk \in \{ (213), (132), (321) \} \\
        0 & {\rm otherwise}
    .\end{cases}
\end{eqnarray}
The Levi-civita symbol is totally anti-symmetric in the sense that permuting indices as $\epsilon_{ijk} = -\epsilon_{jik}$.
This is useful in writing the cross product as
\begin{eqnarray}
    \va*{A} \cross \va*{B} = \vu*{e}_{i} \epsilon_{ijk} A_{j} B_{k}
.\end{eqnarray}
Note the following useful identity:
\begin{eqnarray}
    \epsilon_{ijk} \epsilon_{lmk} = \delta_{il}\delta_{jm} - \delta_{im}\delta_{jl}
.\end{eqnarray}
This can be proven in a number of ways.
The most direct (although at the same time the least satisfying and enlightening) is to just check that the two sides are the same for each pair $(ij)$ and $(lm)$ for $k = 1,2,3$.

(1) Using the framework established above, we have
\begin{eqnarray}
    \eqbox{ \div{[ (\va*{a} \cdot \va*{r}) \va*{b} ]} = \nabla_{i} (\va*{a} \cdot \va*{r}) b_{i} = \nabla_{i} a_{j} r_{j} b_{i} = a_{j} ( \nabla_{i} r_{j} ) b_{i} = a_{j} \delta_{ij} b_{i} = \va*{a} \cdot \va*{b} }
\end{eqnarray}
since $a_{j}$ and $b_{i}$ are constants by assumption.

(2)
\begin{eqnarray}
    \eqbox{ \curl{ [ \va*{a} \cdot \va*{r} ] \va*{b} } = \vu*{e}_{i} \epsilon_{ijk} \nabla_{j} a_{l} r_{l} b_{k} = \vu*{e}_{i} \epsilon_{ijk} a_{l} \delta_{jl} b_{k} = \vu*{e}_{i} \epsilon_{ilk} a_{l}b_{k} = \va*{a} \cross \va*{b} }
.\end{eqnarray}

(3)
\begin{eqnarray}
    \eqbox{ \div{[ \va*{a} \cross \va*{r} ]} = \nabla_{i} \epsilon_{ijk} a_{j} r_{k} = \epsilon_{ijk} a_{j} \delta_{ik} = \epsilon_{kjk} a_{j} = 0 }
.\end{eqnarray}

(4)
\begin{eqnarray}
\eqbox{
\begin{aligned} 
    \curl{ [ \va*{a} \cross \va*{r} ] } &= \vu*{e}_{i} \epsilon_{ijk} \nabla_{j} \epsilon_{klm} a_{l} r_{m} = \vu*{e}_{i} \epsilon_{imk}\epsilon_{lmk} a_{l} \\
    &= \vu*{e}_{i} [ \delta_{il} \delta_{mm} - \delta_{im} \delta_{lm} ] a_{l} = 3 \va*{a} - \va*{a} = 2 \va*{a}
\end{aligned}
}
.\end{eqnarray}

}


\end{document}
