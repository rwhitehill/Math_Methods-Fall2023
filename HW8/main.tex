\def\duedate{\today}
\def\HWnum{8}
\documentclass[10pt,a4paper]{book}

% custom section formatting
\usepackage{titlesec}
\titleformat{\chapter}[display]
{\normalfont\Large\filcenter\sffamily}
{\titlerule[1pt]%
\vspace{1pt}%
\titlerule
\vspace{1pc}%
\LARGE\MakeUppercase{\chaptertitlename} \thechapter}
{1pc}
{\titlerule
\vspace{1pc}%
\Huge}

% appendix handling
\usepackage[toc,page]{appendix}
    
% encoding for file and font
\usepackage[utf8]{inputenc}
\usepackage[T1]{fontenc}

% math formatting/tools
\usepackage{amsmath}
\usepackage{amssymb}
\usepackage{mathtools}
\usepackage[arrowdel]{physics}
\usepackage{dsfont}

\newcommand{\R}{\mathbb{R}}
\newcommand{\Z}{\mathbb{Z}}
\newcommand{\N}{\mathbb{N}}
\newcommand{\Q}{\mathbb{Q}}
\newcommand{\C}{\mathbb{C}}

% unit formatting
\usepackage{siunitx}
\AtBeginDocument{\RenewCommandCopy\qty\SI}

% figure formatting/tools
\usepackage{graphicx}
\usepackage{float}
\usepackage{subcaption}
\usepackage{multirow}
\usepackage{import}
\usepackage{pdfpages}
\usepackage{transparent}
\usepackage{currfile}

\NewDocumentCommand\incfig{O{1} m}{
    \def\svgwidth{#1\textwidth}
    \import{./Figures/\currfiledir}{#2.pdf_tex}
}

\newcommand{\bef}{\begin{figure}[h!tb]\centering}
\newcommand{\eef}{\end{figure}}

\newcommand{\bet}{\begin{table}[h!tb]\centering}
\newcommand{\eet}{\end{table}}

% hyperlink references 
\usepackage{hyperref}
\hypersetup{
    colorlinks=true,
    linkcolor=blue,
    filecolor=magenta,
    urlcolor=cyan,
    pdftitle={Physics 1 Notes},
    pdfauthor={Richard Whitehill},
    pdfpagemode=FullScreen
}
\urlstyle{same}

\newcommand{\eref}[1]{Eq.~(\ref{eq:#1})}
\newcommand{\erefs}[2]{Eqs.~(\ref{eq:#1})--(\ref{eq:#2})}

\newcommand{\fref}[1]{Fig.~(\ref{fig:#1})}
\newcommand{\frefs}[2]{Fig.~(\ref{fig:#1})--(\ref{fig:#2})}

\newcommand{\aref}[1]{Appendix~(\ref{app:#1})}
\newcommand{\sref}[1]{Section~(\ref{sec:#1})}
\newcommand{\srefs}[2]{Sections~(\ref{sec:#1})-(\ref{sec:#2})}

\newcommand{\tref}[1]{Table~(\ref{tab:#1})}
\newcommand{\trefs}[2]{Table~(\ref{tab:#1})--(\ref{tab:#2})}

% tcolorbox formatting/definitions
\usepackage[most]{tcolorbox}
\usepackage{xcolor}
\usepackage{xifthen}
\usepackage{parskip}

\definecolor{peach}{rgb}{1.0,0.8,0.64}

\DeclareTColorBox[auto counter, number within=chapter]{defbox}{O{}}{
    enhanced,
    boxrule=0pt,
    frame hidden,
    borderline west={4pt}{0pt}{green!50!black},
    colback=green!5,
    before upper=\textbf{Definition \thetcbcounter \ifthenelse{\isempty{#1}}{}{: #1} \\ },
    sharp corners
}

\newcommand*{\eqbox}{\tcboxmath[
    enhanced,
    colback=black!10!white,
    colframe=black,
    sharp corners,
    size=fbox,
    boxsep=8pt,
    boxrule=1pt
]}

\newtcolorbox[auto counter, number within=chapter]{exbox}{
    parbox=false,
    breakable,
    enhanced,
    sharp corners,
    boxrule=1pt,
    colback=white,
    colframe=black,
    before upper= \textbf{Example \thetcbcounter:}\,,
    before lower= \textbf{Solution:}\,,
    segmentation hidden
}

\newtcolorbox{resbox}{
    enhanced,
    colback=black!10!white,
    colframe=black,
    boxrule=1pt,
    boxsep=0pt,
    top=2pt,
    ams nodisplayskip,
    sharp corners
}


\begin{document}

\prob{1}{

Find the Laplace transform of the following function:
\begin{eqnarray}
    I(t) = t^{n} e^{-at}, \quad a > 0 ~{\rm and~even}~n
.\end{eqnarray}

}

\sol{

The Laplace transform of a function $f(t)$ is generally given as
\begin{eqnarray}
    \mathcal{L}\{ f(t) \} = \int_{0}^{\infty} f(t) e^{-st} \dd{t}
,\end{eqnarray}
where the values of $s$ are such that the integral is convergent.
Thus,
\begin{eqnarray}
\begin{aligned}
    \mathcal{L}\{ I(t) \} &= \int_{0}^{\infty} t^{n} e^{-(a+s)t} \dd{t} = (-1)^{n} \dv[n]{(a+s)} \int_{0}^{\infty} e^{-(a+s)t} \dd{t} \\
                          &= (-1)^{n} \dv[n]{(a+s)} \frac{1}{a+s} = \frac{n!}{(a+s)^{n+1}}
.\end{aligned}
\end{eqnarray}
This matches the result we have in our library.
That is, $\mathcal{L}\{ t^{n} \} = n!/s^{n+1}$ and for a general function $f(t)$, $\mathcal{L}\{ f(t)e^{-at} \} = F(s + a)$.
Collecting these results, we end up with the same solution, which is that
\begin{eqnarray}
    \mathcal{L}\{ t^{n} e^{-at} \} = \frac{n!}{(s + a)^{n+1}}
.\end{eqnarray}

}


\prob{2}{

Solve the following equation by the Laplace transform
\begin{eqnarray}
    \ddot{y} + 2\lambda \dot{y} + \omega_0^2 y = 0
,\end{eqnarray}
where $y(0) = 0$ and $\dot{y}(0) = v$.

}

\sol{

If we take the Laplace transform of the equation\footnote{Strictly speaking, this means that if we have a differential equation $D f = g$, where $D$ is a linear differential operator, then its ``Laplace transform'' is $\mathcal{L}\{ D f \} = G(s)$.}, we find
\begin{eqnarray}
    \mathcal{L}\{ \ddot{y} + 2 \lambda \dot{y} + \omega_0^2 y \} = [ s^2 Y(s) - sv ] + 2 \lambda [ sY(s) ] + \omega_0^2 Y(s) = 0
.\end{eqnarray}
Solving for $Y(s)$ gives
\begin{eqnarray}
    Y(s) = \frac{sv}{s^2 + 2\lambda s + \omega_0^2} = \frac{s v}{(s + \lambda)^2 + (\omega_0^2 - \lambda^2)}
.\end{eqnarray}
From our ``library'' we have
\begin{align}
    \mathcal{L}\{ e^{-at}\cos{bt} \} &= \frac{s+a}{(s+a)^2 + b^2} \\
    \mathcal{L}\{ e^{-at}\sin{bt} \} &= \frac{b}{(s+a)^2 + b^2}
.\end{align}
Thus, if we write $a = \lambda$ and $b = \sqrt{\omega_0^2 - \lambda^2}$, then
\begin{eqnarray}
    Y(s) = v\Bigg[ \frac{s+a}{(s+a)^2 + b^2} - \frac{a}{(s+a)^2 + b^2} \Bigg]
.\end{eqnarray}


}


\prob{3}{

A unit vector $\vu*{n}$ makes angles $\theta$ and $\alpha$ with the Cartesian axes $z$ and $x$, respectively, and a unit vector $\vu*{n}'$ makes angles $\theta'$ and $\alpha'$ with $z$ and $x$, respectively.
Find $\cos{\varphi}$, where $\varphi$ is the angle between $\vu*{n}$ and $\vu*{n}'$.

}

\sol{}


\prob{4}{

Find a scalar function $\varphi(r)$ of $r = |\va*{r}|$ which satisfies the equation
\begin{eqnarray}
    \div{[ \varphi(r) \va*{r} ]} = 0
.\end{eqnarray}

}

\sol{}


\prob{5}{

    Calculate the following: (1) $\div{[(\va*{a} \cdot \va*{r}) \va*{b}]}$, (2) $\curl{[(\va*{a} \cdot \va*{r})\va*{b}]}$, (3) $\div{\va*{a} \cross \va*{r}}$, (4) $\curl{(\va*{a} \cross \va*{r})}$, (5) $\div{[\va*{r} \cross (\va*{a} \cross \va*{r})]}$, where $\va*{a}$ and $\va*{b}$ are constant vectors.

}

\sol{}


\end{document}
