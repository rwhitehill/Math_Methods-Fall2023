\def\duedate{\today}
\def\HWnum{11}
\documentclass[10pt,a4paper]{book}

% custom section formatting
\usepackage{titlesec}
\titleformat{\chapter}[display]
{\normalfont\Large\filcenter\sffamily}
{\titlerule[1pt]%
\vspace{1pt}%
\titlerule
\vspace{1pc}%
\LARGE\MakeUppercase{\chaptertitlename} \thechapter}
{1pc}
{\titlerule
\vspace{1pc}%
\Huge}

% appendix handling
\usepackage[toc,page]{appendix}
    
% encoding for file and font
\usepackage[utf8]{inputenc}
\usepackage[T1]{fontenc}

% math formatting/tools
\usepackage{amsmath}
\usepackage{amssymb}
\usepackage{mathtools}
\usepackage[arrowdel]{physics}
\usepackage{dsfont}

\newcommand{\R}{\mathbb{R}}
\newcommand{\Z}{\mathbb{Z}}
\newcommand{\N}{\mathbb{N}}
\newcommand{\Q}{\mathbb{Q}}
\newcommand{\C}{\mathbb{C}}

% unit formatting
\usepackage{siunitx}
\AtBeginDocument{\RenewCommandCopy\qty\SI}

% figure formatting/tools
\usepackage{graphicx}
\usepackage{float}
\usepackage{subcaption}
\usepackage{multirow}
\usepackage{import}
\usepackage{pdfpages}
\usepackage{transparent}
\usepackage{currfile}

\NewDocumentCommand\incfig{O{1} m}{
    \def\svgwidth{#1\textwidth}
    \import{./Figures/\currfiledir}{#2.pdf_tex}
}

\newcommand{\bef}{\begin{figure}[h!tb]\centering}
\newcommand{\eef}{\end{figure}}

\newcommand{\bet}{\begin{table}[h!tb]\centering}
\newcommand{\eet}{\end{table}}

% hyperlink references 
\usepackage{hyperref}
\hypersetup{
    colorlinks=true,
    linkcolor=blue,
    filecolor=magenta,
    urlcolor=cyan,
    pdftitle={Physics 1 Notes},
    pdfauthor={Richard Whitehill},
    pdfpagemode=FullScreen
}
\urlstyle{same}

\newcommand{\eref}[1]{Eq.~(\ref{eq:#1})}
\newcommand{\erefs}[2]{Eqs.~(\ref{eq:#1})--(\ref{eq:#2})}

\newcommand{\fref}[1]{Fig.~(\ref{fig:#1})}
\newcommand{\frefs}[2]{Fig.~(\ref{fig:#1})--(\ref{fig:#2})}

\newcommand{\aref}[1]{Appendix~(\ref{app:#1})}
\newcommand{\sref}[1]{Section~(\ref{sec:#1})}
\newcommand{\srefs}[2]{Sections~(\ref{sec:#1})-(\ref{sec:#2})}

\newcommand{\tref}[1]{Table~(\ref{tab:#1})}
\newcommand{\trefs}[2]{Table~(\ref{tab:#1})--(\ref{tab:#2})}

% tcolorbox formatting/definitions
\usepackage[most]{tcolorbox}
\usepackage{xcolor}
\usepackage{xifthen}
\usepackage{parskip}

\definecolor{peach}{rgb}{1.0,0.8,0.64}

\DeclareTColorBox[auto counter, number within=chapter]{defbox}{O{}}{
    enhanced,
    boxrule=0pt,
    frame hidden,
    borderline west={4pt}{0pt}{green!50!black},
    colback=green!5,
    before upper=\textbf{Definition \thetcbcounter \ifthenelse{\isempty{#1}}{}{: #1} \\ },
    sharp corners
}

\newcommand*{\eqbox}{\tcboxmath[
    enhanced,
    colback=black!10!white,
    colframe=black,
    sharp corners,
    size=fbox,
    boxsep=8pt,
    boxrule=1pt
]}

\newtcolorbox[auto counter, number within=chapter]{exbox}{
    parbox=false,
    breakable,
    enhanced,
    sharp corners,
    boxrule=1pt,
    colback=white,
    colframe=black,
    before upper= \textbf{Example \thetcbcounter:}\,,
    before lower= \textbf{Solution:}\,,
    segmentation hidden
}

\newtcolorbox{resbox}{
    enhanced,
    colback=black!10!white,
    colframe=black,
    boxrule=1pt,
    boxsep=0pt,
    top=2pt,
    ams nodisplayskip,
    sharp corners
}


\begin{document}

\prob{1}{

Write down a general solution of the Tomas-Fermi equation
\begin{eqnarray}
    \laplacian{\varphi} - \lambda^{-2} \varphi = f(\va*{r})
\end{eqnarray}
with the boundary condition $\varphi(x,y,0) = 0$ at a conducting plane $z = 0$ and $\varphi(x,y,z) \rightarrow 0$ at $x,y \rightarrow \infty$.
The function $f(\va*{r})$ tends to zero at infinity.

}

\sol{

In class we found the Green function for the Tomas-Fermi problem in $\mathbb{R}^3$ (i.e. the boundary conditions are just that $G \rightarrow 0$ as $\va*{r}' \rightarrow \infty$) as
\begin{eqnarray}
    G_{\infty}(x,y,z;x',y',z') = -\frac{e^{-|\va*{r}-\va*{r}'|/\lambda}}{4 \pi |\va*{r} - \va*{r}'|}
.\end{eqnarray}
We can solve the problem in the half-space $z \geq 0$ with the Green's function
\begin{eqnarray}
    G(x,y,z;x',y',z') = G_{\infty}(x,y,z;x',y',z') - G_{\infty}(x,y,z;x',y',-z')
.\end{eqnarray}
Essentially, we have used the method of images where we placed an ``image source'' reflected over the $xy$-plane from the point source described by $G_{\infty}$.
Notice that on the plane we then have $G(x,y,z;x',y',0) = 0$ and $G(x,y,z;x'\rightarrow \infty,y'\rightarrow \infty, z') = 0$ automatically since as $x',y' \rightarrow \infty$ we have $|\va*{r} - \va*{r}'| \rightarrow \infty$ and therefore $G_{\infty} \rightarrow 0$ under this condition by construction.
The solution is then
\begin{eqnarray}
    \eqbox{ \varphi(\va*{r}) = -\frac{1}{4 \pi} \int_{-\infty}^{\infty} \dd{x'} \int_{-\infty}^{\infty} \dd{y'} \int_{0}^{\infty} \dd{z'} G(x,y,z;x',y',z') f(\va*{r}') }
.\end{eqnarray}


}


\prob{2}{

Transverse displacements $u(x,t)$ of a string of length $L$ with fixed ends are desribed by the wave equation
\begin{eqnarray}
    \pdv[2]{u}{t} = s^2 \pdv[2]{u}{x}
\end{eqnarray}
with the boundary condition $u(0,t) = u(L,t) = 0$.
At $t = 0$ a hammer of width $a < L$ hits the string which was initially at rest, $u(x,0) = 0$.
The impact caused an instantaneous velocity $v_0$ in the central region of the string:
\begin{eqnarray}
    \dot{u}(x,0) = v_0
\end{eqnarray}
for $|x - L/2| < a$ and $\dot{u}(x,0) = 0$ for $|x - L/2| > a$.
Find the full solution $u(x,t)$ at $t > 0$.

}

\sol{

We solved the one-dimensional wave equation on $x \in [0,L], t \geq 0$ for generic initial conditions and fixed endpoints $u(0,t) = u(L,t) = 0$ as boundary conditions as
\begin{eqnarray}
    u(x,t) = \sum_{n=1}^{\infty} \Big[ A_{n} \cos( \frac{n \pi s t}{L} ) + B_{n} \sin( \frac{n \pi s t}{L} ) \Big] \sin( \frac{n \pi x}{L} )
.\end{eqnarray}
Notice that at $t = 0$, the position of the string is
\begin{eqnarray}
    u(x,0) = \sum_{n=1}^{\infty} A_{n} \sin(\frac{n \pi x}{L}) = 0
.\end{eqnarray}
Since this must be true for all $x$ and the different modes are orthogonal, we must have $A_{n} = 0$ for all $n$.
Next, we have the initial velocity of the string
\begin{eqnarray}
    \dot{u}(x,0) = \sum_{n=1}^{\infty} B_{n} \frac{n \pi s}{L} \sin(\frac{n \pi x}{L}) = v_0 \theta( |x - L/2| < a )
.\end{eqnarray}
We can use the orthogonality relations between sine modes, which gives
\begin{eqnarray}
    B_{n} = \frac{L}{n \pi s} \frac{2}{L} \int_{0}^{L} v_0 \theta(|x - L/2| < a) \dd{x} = \frac{4 v_0 a}{n \pi s}
.\end{eqnarray}
Putting this into the expansion for $u$, we have
\begin{eqnarray}
    \eqbox{ u(x,t) = \frac{4 v_0 a}{\pi s} \sum_{n=1}^{\infty} \frac{1}{n} \sin( \frac{n \pi s t}{L} ) \sin(\frac{n \pi x}{L}) }
.\end{eqnarray}

}


\prob{3}{

Solve the Laplace equation
\begin{eqnarray}
    \laplacian u = 0
\end{eqnarray}
for a circle of radius $a$ with the boundary condition $u(a,\varphi) = u_0 \sin^3{\varphi}$.

\vspace{1em}

(a) Show that the solution $u(r,\varphi)$ inside the circle $r \leq a$ can be found by the separation of variables and calculate the coefficients $A_{n}$ and $B_{n}$:
\begin{eqnarray}
    u(r,\varphi) = \sum_{n=0}^{\infty} r^{n} \Big[ A_{n} \sin{n \varphi} + B_{n} \cos{n \varphi} \Big]
.\end{eqnarray}

(b) Show that the solution $u(r,\varphi)$ outside the circle $r > a$ can be found by the separation of variables and calculate the coefficients $A_{n}$ and $B_{n}$:
\begin{eqnarray}
    u(r,\varphi) = \sum_{n=0}^{\infty} r^{-n} \Big[ A_{n} \sin{n \varphi} + B_{n} \cos{n \varphi} \Big]
.\end{eqnarray}

}

\sol{

The Laplacian in polar coordinates (or cylindrical coordinates with translation symmetry along the $z$-axis) is
\begin{eqnarray}
    \laplacian = \frac{1}{r} \pdv{r} \Big( r \pdv{u}{r} \Big) + \frac{1}{r^2} \pdv[2]{u}{\varphi}
.\end{eqnarray}
We can use separation of variables and write $u(r,\varphi) = R(r) T(\varphi)$, which gives
\begin{eqnarray}
    \frac{r}{R} \dv{r} \Big( r \dv{R}{r} \Big) + \frac{1}{T} \dv[2]{T}{\varphi} = 0
.\end{eqnarray}
Both terms must be constant with respect to $r,\varphi$, and since $T$ must be cyclic (i.e. $u$ must be single-valued), we choose
\begin{eqnarray}
    \dv[2]{T}{\phi} = -n^2 T
,\end{eqnarray}
which has solutions $\sin{n \varphi}$ and $\cos{n \varphi}$.
Putting this into the separated laplace's equation, we find a differential equation for the radial part of $u$ as
\begin{eqnarray}
    r \dv{r} \Big( r \dv{R}{r} \Big) - n^2 R = 0
.\end{eqnarray}
Let us propose a solution of the form $R(r) = \sum_{m=0}^{\infty} a_{m} r^{m+s}$.
We then find that
\begin{eqnarray}
    \sum_{n=0}^{\infty} a_{m} [(m+s)^2 - n^2] r^{m+s} = 0
.\end{eqnarray}
We then find that either $a_{m} = 0$ or $m+s = \pm n$ makes each term zero.
Thus, we only have two terms for $R$ at a fixed $n$ such that
\begin{eqnarray}
    R_{n}(r) = a_{n} r^{n} + b_{n} r^{-n}
.\end{eqnarray}
A general solution for Laplace's equation is
\begin{eqnarray}
    u(r,\varphi) = \sum_{n=0}^{\infty} \Big[ a_{n} r^{n} + b_{n} r^{-n} \Big] \Big[ A_{n} \sin{n \varphi} + B_{n} \cos{n \varphi} \Big]
.\end{eqnarray}

(a) If we wish to consider the solution of Laplace's equation with boundary conditions at $r = a$ for $r < a$, we must have $b_{n} = 0$ since our solution should be regular at $r = 0$.
Thus,
\begin{eqnarray}
    u(r,\varphi) = \sum_{n=0}^{\infty} r^{n} \Big[ A_{n} \sin{n \varphi} + B_{n} \cos{n \varphi} \Big]
.\end{eqnarray}
Next, we find that $\sin^3{\varphi} = [ 3 \sin{\varphi} - \sin{3\varphi} ]/4$, meaning that our solution for $r < a$ must be
\begin{eqnarray}
    \eqbox{ u(r,\varphi) = \frac{u_0}{4} \Big[ \frac{3r}{a} \sin{\varphi} - \Big( \frac{r}{a} \Big)^{3} \sin{3 \varphi} \Big] }
.\end{eqnarray}

(b) Now, if we consider $r > a$, we similarly must have $a_{n} = 0$ so that our solution $u \rightarrow 0$ as $r \rightarrow \infty$, and thus for $r > a$
\begin{eqnarray}
    \eqbox{ u(r,\varphi) = \frac{u_0}{4} \Big[ \frac{3a}{r} \sin{\varphi} - \Big( \frac{a}{r} \Big)^3 \sin{3 \varphi} \Big] }
.\end{eqnarray}

}



\end{document}
