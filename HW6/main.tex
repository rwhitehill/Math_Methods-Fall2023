\def\duedate{\today}
\def\HWnum{6}
\documentclass[10pt,a4paper]{book}

% custom section formatting
\usepackage{titlesec}
\titleformat{\chapter}[display]
{\normalfont\Large\filcenter\sffamily}
{\titlerule[1pt]%
\vspace{1pt}%
\titlerule
\vspace{1pc}%
\LARGE\MakeUppercase{\chaptertitlename} \thechapter}
{1pc}
{\titlerule
\vspace{1pc}%
\Huge}

% appendix handling
\usepackage[toc,page]{appendix}
    
% encoding for file and font
\usepackage[utf8]{inputenc}
\usepackage[T1]{fontenc}

% math formatting/tools
\usepackage{amsmath}
\usepackage{amssymb}
\usepackage{mathtools}
\usepackage[arrowdel]{physics}
\usepackage{dsfont}

\newcommand{\R}{\mathbb{R}}
\newcommand{\Z}{\mathbb{Z}}
\newcommand{\N}{\mathbb{N}}
\newcommand{\Q}{\mathbb{Q}}
\newcommand{\C}{\mathbb{C}}

% unit formatting
\usepackage{siunitx}
\AtBeginDocument{\RenewCommandCopy\qty\SI}

% figure formatting/tools
\usepackage{graphicx}
\usepackage{float}
\usepackage{subcaption}
\usepackage{multirow}
\usepackage{import}
\usepackage{pdfpages}
\usepackage{transparent}
\usepackage{currfile}

\NewDocumentCommand\incfig{O{1} m}{
    \def\svgwidth{#1\textwidth}
    \import{./Figures/\currfiledir}{#2.pdf_tex}
}

\newcommand{\bef}{\begin{figure}[h!tb]\centering}
\newcommand{\eef}{\end{figure}}

\newcommand{\bet}{\begin{table}[h!tb]\centering}
\newcommand{\eet}{\end{table}}

% hyperlink references 
\usepackage{hyperref}
\hypersetup{
    colorlinks=true,
    linkcolor=blue,
    filecolor=magenta,
    urlcolor=cyan,
    pdftitle={Physics 1 Notes},
    pdfauthor={Richard Whitehill},
    pdfpagemode=FullScreen
}
\urlstyle{same}

\newcommand{\eref}[1]{Eq.~(\ref{eq:#1})}
\newcommand{\erefs}[2]{Eqs.~(\ref{eq:#1})--(\ref{eq:#2})}

\newcommand{\fref}[1]{Fig.~(\ref{fig:#1})}
\newcommand{\frefs}[2]{Fig.~(\ref{fig:#1})--(\ref{fig:#2})}

\newcommand{\aref}[1]{Appendix~(\ref{app:#1})}
\newcommand{\sref}[1]{Section~(\ref{sec:#1})}
\newcommand{\srefs}[2]{Sections~(\ref{sec:#1})-(\ref{sec:#2})}

\newcommand{\tref}[1]{Table~(\ref{tab:#1})}
\newcommand{\trefs}[2]{Table~(\ref{tab:#1})--(\ref{tab:#2})}

% tcolorbox formatting/definitions
\usepackage[most]{tcolorbox}
\usepackage{xcolor}
\usepackage{xifthen}
\usepackage{parskip}

\definecolor{peach}{rgb}{1.0,0.8,0.64}

\DeclareTColorBox[auto counter, number within=chapter]{defbox}{O{}}{
    enhanced,
    boxrule=0pt,
    frame hidden,
    borderline west={4pt}{0pt}{green!50!black},
    colback=green!5,
    before upper=\textbf{Definition \thetcbcounter \ifthenelse{\isempty{#1}}{}{: #1} \\ },
    sharp corners
}

\newcommand*{\eqbox}{\tcboxmath[
    enhanced,
    colback=black!10!white,
    colframe=black,
    sharp corners,
    size=fbox,
    boxsep=8pt,
    boxrule=1pt
]}

\newtcolorbox[auto counter, number within=chapter]{exbox}{
    parbox=false,
    breakable,
    enhanced,
    sharp corners,
    boxrule=1pt,
    colback=white,
    colframe=black,
    before upper= \textbf{Example \thetcbcounter:}\,,
    before lower= \textbf{Solution:}\,,
    segmentation hidden
}

\newtcolorbox{resbox}{
    enhanced,
    colback=black!10!white,
    colframe=black,
    boxrule=1pt,
    boxsep=0pt,
    top=2pt,
    ams nodisplayskip,
    sharp corners
}


\begin{document}

\prob{1}{

Calculate the following integral:
\begin{eqnarray}
    I_1 = \int_{0}^{\infty} \frac{x^{1/4} \dd{x}}{a^2 + x^2}
.\end{eqnarray}

}

\sol{

Let us make use of the substitution $x = ay$, which allows us to write
\begin{eqnarray}
    I_1 = \frac{a^{1/4} a}{a^2} \int_{0}^{\infty} \frac{y^{1/4}}{1 + y^2} \dd{y} = a^{-3/4} \int_{0}^{\infty} \frac{y^{1/4}}{1 + y^2} \dd{y}
.\end{eqnarray}
Next, let us write $y = e^{t/2}$, which gives
\begin{eqnarray}
    \eqbox{ I_1 = \frac{1}{2} a^{-3/4} \int_{-\infty}^{\infty} \frac{e^{t/8} e^{t/2}}{1 + e^{t}} \dd{t} = \frac{1}{2} a^{-3/4} \int_{-\infty}^{\infty} \frac{e^{5t/8}}{1 + e^{t}} \dd{t} = \frac{\pi}{2a^{3/4}\sin(5\pi/8)} }
.\end{eqnarray}

}


\prob{2}{

Calculate the following integral at $s\gg 1$:
\begin{eqnarray}
    I_2 = \int_{1}^{\infty} e^{s(x - x^2)} \dd{x}
.\end{eqnarray}

}

\sol{

Notice that we can write $x^2 - x = (x - 1) + (x-1)^2$.
If $s$ is sufficiently large, then $\exp[s(x-x^2)] \approx \exp[-s(x-1)]$, meaning
\begin{eqnarray}
    \eqbox{ I_2 \approx e^{s} \int_{1}^{\infty} e^{-sx} \dd{x} = \frac{1}{s} }
.\end{eqnarray}

}


\prob{3}{

Calculate the following integral at $s \gg 1$:
\begin{eqnarray}
    I_3 = \int_{0}^{\infty} \exp(- \frac{s^2}{x} - x) \dd{x}
.\end{eqnarray}

}

\sol{

Let $f(x;s) = s/x + x/s$.
Notice then that $f'(x;s) = -s/x^2 + 1/s$, implying that $f$ has a minimum at $x = s$ in the integration region, allowing us to write
\begin{eqnarray}
    f(x;s) = f(s) + \frac{f''(s)}{2} (x - s)^2 + \ldots = 2 + \frac{1}{s^2} (x-s)^2 + \ldots
\end{eqnarray}
which makes
\begin{eqnarray}
    I_3 \approx e^{-2s} \int_{0}^{\infty} e^{-(x-s)^2/s} \dd{x}
.\end{eqnarray}

{\color{red} This is wrong -- the expansion for $f(x;s)$ actually only converges for $|x - s| < s$ -- need to think a little bit more about this}

}


\prob{4}{

Calculate the following integral at $s \gg 1$:
\begin{eqnarray}
    I_4 = \int_{0}^{\infty} x^{\alpha} e^{-sx^2} \dd{x}
,\end{eqnarray}
where $\alpha > 0$ is an arbitrary (not necessarily integer) number.

}

\sol{

Notice that we can write $x^{\alpha} = e^{\alpha \ln{x}}$ such that our integral becomes
\begin{eqnarray}
    I_4 = \int_{0}^{\infty} e^{-s(x^2 - \alpha \ln{x})} \dd{x}
,\end{eqnarray}
and if we denote $f(x) = x^2 - \alpha \ln{x}$, observe that $f$ attains a minimum at $x = \sqrt{\alpha/2}$, which is positive since $\alpha > 0$.
Hence, we can write
\begin{eqnarray}
    \eqbox{
    \begin{aligned}
        I_4 &\approx e^{-\frac{s \alpha}{2}[1 - \ln(\alpha/2)]}\int_{-\infty}^{\infty} e^{-2s \big[x - \sqrt{\alpha/2} \big]^2} \dd{x} \\
        &= e^{-\frac{s\alpha}{2} [1 - \ln(\alpha/2)]} \sqrt{\frac{\pi}{2s}} = \Big( \frac{\alpha}{2e} \Big)^{s \alpha/2} \sqrt{\frac{\pi}{2s}}
    .\end{aligned}
    }
\end{eqnarray}

}



\end{document}
