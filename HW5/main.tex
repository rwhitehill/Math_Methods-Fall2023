\def\duedate{\today}
\def\HWnum{5}
\input{../preamble.tex}

\begin{document}

\prob{1}{

Find the branch points of this function:
\begin{eqnarray}
   f(z) = \sqrt{z^2 + 2z - 1}
.\end{eqnarray}
What branch cuts can make this function single-valued?

}

\sol{

We can write
\begin{eqnarray}
    f(z) = \sqrt{\Big[z - \Big(-1 + \sqrt{2} \Big)\Big] \Big[z - \Big(-1 - \sqrt{2} \Big) \Big]}
.\end{eqnarray}
Thus, there are branch points at $z = -1 \pm \sqrt{2}$.
We can make a couple different branch cuts to make this function single valued such as those shown in \fref{prob1}.
The first is just connecting the two points via a line on the real axis, and the second is the lines $(\infty,-1 - \sqrt{2}]$ and $[-1 + \sqrt{2},\infty)$.

\begin{figure}[h!]
    \centering
    %\includegraphics[width=\textwidth]{}
    \caption{}
    \label{fig:prob1}
\end{figure}

}


\prob{2}{

Find the residues and all isolated singularities of the function
\begin{eqnarray}
    I_{2}(z) = \tan{z}
.\end{eqnarray}

}

\sol{

Observe that
\begin{eqnarray}
    \tan{z} = \frac{\sin{z}}{\cos{z}} = \dfrac{\sin{z}}{\prod_{n=0}^{\infty} \Big[ 1 - \frac{4 z^2}{\pi^2(2n+1)^2} \Big]}
.\end{eqnarray}
Clearly, the function becomes singular at $z = \pm(n + 1/2)\pi$ for $n = 0,1,2,\ldots$, and furthermore, observe that these are all isolated singularities and simple poles.
Notice that in the neighborhood of these poles (let $z_{n} = (n + 1/2)\pi$ where $n = 0,\pm 1,\pm 2,\ldots$) we can write
\begin{eqnarray}
    \begin{aligned}
        \cos{z} &= \cos{(w + z_{n})} = \cos{w}\cos{z_{n}} - \sin{w} \sin{z_{n}} = (-1)^{n+1} \sin{w} \\
                &= (-1)^{n+1} \sum_{k=0}^{\infty} \frac{(-1)^{k}}{(2k+1)!} (z - z_{n})^{2k+1} \\
                &= (z - z_{n}) \Bigg[ (-1)^{n+1} \sum_{k=0}^{\infty} \frac{(-1)^{k}}{(2k+1)!} (z - z_{n})^{2k} \Bigg]
    ,\end{aligned}
\end{eqnarray}
where $w = z - z_{n}$ and we have expanded around $w = 0$.
Thus, we find
\begin{eqnarray}
    \Res{\tan{z}}|_{z = z_{n}} = \frac{\sin{z_{n}}}{(-1)^{n+1}} = -1
.\end{eqnarray}

Alternatively, the ``formula'' for this kind of situation is that $\Res{\tan{z}}|_{z = z_{n}} = \\ \sin{z_{n}}/[-\sin{z_{n}}] = -1$.
Interestingly, this result is independent of where the singularity is located.

}


\prob{3}{

Calculate the following real integral using the Cauchy theorem:
\begin{eqnarray}
    I_{3} = \int_{0}^{2\pi} \frac{\dd{x}}{2 + \cos^2{x}}
.\end{eqnarray}

}

\sol{

Let us write $\cos^2{x} = (1 + \cos{2 x})/2$ such that
\begin{eqnarray}
    \eqbox{ I_{3} = 2 \int_{0}^{2\pi} \frac{\dd{x}}{5 + \cos{2x}} = \frac{1}{5} \int_{0}^{2 \pi} \frac{\dd{y}}{1 + \frac{1}{5}\cos{y}} = \frac{1}{5} \frac{2 \pi}{\sqrt{1 - (1/5)^2}} = \frac{\pi}{\sqrt{6}} }
,\end{eqnarray}
where we have used the substitution $y = 2x$ and the result
\begin{eqnarray}
    \int_{0}^{2 \pi} \frac{\dd{x}}{1 + \epsilon \cos{x}} = \frac{2 \pi}{\sqrt{1 - \epsilon^2}}
\end{eqnarray}
for $|\epsilon| < 1$

}


\prob{4}{

Calculate the following real integral using the Cauchy theorem:
\begin{eqnarray}
    I_{4} = \int_{-\infty}^{\infty} \frac{\dd{x}}{1 + x^{4}}
.\end{eqnarray}

}

\sol{

Notice that $f(z) = 1/(1 + z^4) \approx \frac{1}{|z|^{4}} \rightarrow 0$ as $|z| \rightarrow \infty$, and therefore we can write
\begin{eqnarray}
    I_{4} = \oint_{C} \frac{\dd{z}}{1 + z^{4}}
,\end{eqnarray}
where $C = (-\infty,\infty) + \Gamma$, where $\Gamma$ is the path along semi-circle in either the upper or lower half plane.
\begin{eqnarray}
    1 + z^{4} = (z^2 + i)(z^2 - i) = (z - i\sqrt{i})(z + i\sqrt{i})(z - \sqrt{i})(z + \sqrt{i})
.\end{eqnarray}
Note that $\sqrt{i} = e^{i \pi/4} = \cos{(\pi /4)} + i \sin{(\pi / 4)} = (1 + i)/\sqrt{2}$.
There are then two roots in the upper half plane ($z = \sqrt{i},i\sqrt{i}$) and another two in the lower half plane ($z = -\sqrt{i},-i\sqrt{i}$), so we choose $\Gamma$ in the upper half plane since it is oriented in a way compatible inherently with Cauchy's theorem (although the contour in the lower half plane just comes in with a minus sign which is not prohibitively more complicated).
Thus, the integral
\begin{eqnarray}
    \begin{aligned}
        I_{4} &= 2 \pi i \Bigg[ \frac{1}{(z^2 + i)(z + \sqrt{i})}\Big|_{z = \sqrt{i}} + \frac{1}{(z + i\sqrt{i})(z^2 - i)}\Big|_{z = i\sqrt{i}} \Bigg] \\
              &= 2 \pi i \Bigg[ \frac{1}{(2i)(2\sqrt{i})} + \frac{1}{(2i\sqrt{i})(-2i)} \Bigg] \\
              &= \pi \Bigg[ \frac{1}{2\sqrt{i}} - \frac{1}{2i\sqrt{i}} \Bigg] \\
              &= \pi \Big( \frac{-1 + i}{2i\sqrt{i}} \Big) = \eqbox{ \frac{\pi}{\sqrt{2}} }
    .\end{aligned}
\end{eqnarray}

}


\prob{5}{

Calculate the following real integral using the Cauchy theorem:
\begin{eqnarray}
    I_{5}(a) = \int_{0}^{\infty} \frac{x \sin{ax}}{b^2 + x^2} \dd{x}
,\end{eqnarray}
where $a > 0$.

}

\sol{

Let us write
\begin{eqnarray}
    I_{5} = \frac{1}{2} \Im{ \int_{-\infty}^{\infty} \frac{x e^{iax}}{b^2 + x^2} \dd{x} }
.\end{eqnarray}
Since $f(z) = z/(b^2 + z^2) \rightarrow 0$ as $|z| \rightarrow \infty$ and $a > 0$, we can write
\begin{eqnarray}
    \int_{-\infty}^{\infty} \frac{x e^{iax}}{b^2 + x^2} \dd{x} = \oint_{C} \frac{z e^{iaz}}{b^2 + z^2} \dd{z}
,\end{eqnarray}
where $C = (-\infty,\infty) + \Gamma$, where $\Gamma$ is the half-circle in the upper half-plane, which encloses the simple pole of the integrand $z = ib$.
Hence,
\begin{eqnarray}
    I_{5} = \frac{1}{2} \Im{ 2 \pi i \frac{ib e^{ia(ib)}}{2ib} } = \frac{\pi}{2} \Im{ i e^{-ab} } = \frac{\pi}{2} e^{-ab}
.\end{eqnarray}
Note that the result does not depend on the sign of $b$, so we append the absolute value sign to $b$ in our expression and arrive at
\begin{eqnarray}
    \eqbox{ I_{5} = \frac{\pi}{2} e^{-a|b|} }
.\end{eqnarray}

We can check our answer as follows.
Note that 
\begin{eqnarray}
    \begin{aligned}
        I_{5} &= \frac{1}{2} \Im{ -i \pdv{a} \int_{-\infty}^{\infty} \frac{e^{iax}}{b^2 + x^2} \dd{x} } = \frac{1}{2} \Im{ -i \pdv{a} \frac{\pi}{|b|} e^{-a|b|} } \\
              &= \frac{1}{2} \Im{ i \pi e^{-a|b|} } = \frac{\pi}{2} e^{-a|b|}
    ,\end{aligned}
\end{eqnarray}
which is exactly the result quoted above.


}




\end{document}
