\def\duedate{\today}
\def\HWnum{10}
\documentclass[10pt,a4paper]{book}

% custom section formatting
\usepackage{titlesec}
\titleformat{\chapter}[display]
{\normalfont\Large\filcenter\sffamily}
{\titlerule[1pt]%
\vspace{1pt}%
\titlerule
\vspace{1pc}%
\LARGE\MakeUppercase{\chaptertitlename} \thechapter}
{1pc}
{\titlerule
\vspace{1pc}%
\Huge}

% appendix handling
\usepackage[toc,page]{appendix}
    
% encoding for file and font
\usepackage[utf8]{inputenc}
\usepackage[T1]{fontenc}

% math formatting/tools
\usepackage{amsmath}
\usepackage{amssymb}
\usepackage{mathtools}
\usepackage[arrowdel]{physics}
\usepackage{dsfont}

\newcommand{\R}{\mathbb{R}}
\newcommand{\Z}{\mathbb{Z}}
\newcommand{\N}{\mathbb{N}}
\newcommand{\Q}{\mathbb{Q}}
\newcommand{\C}{\mathbb{C}}

% unit formatting
\usepackage{siunitx}
\AtBeginDocument{\RenewCommandCopy\qty\SI}

% figure formatting/tools
\usepackage{graphicx}
\usepackage{float}
\usepackage{subcaption}
\usepackage{multirow}
\usepackage{import}
\usepackage{pdfpages}
\usepackage{transparent}
\usepackage{currfile}

\NewDocumentCommand\incfig{O{1} m}{
    \def\svgwidth{#1\textwidth}
    \import{./Figures/\currfiledir}{#2.pdf_tex}
}

\newcommand{\bef}{\begin{figure}[h!tb]\centering}
\newcommand{\eef}{\end{figure}}

\newcommand{\bet}{\begin{table}[h!tb]\centering}
\newcommand{\eet}{\end{table}}

% hyperlink references 
\usepackage{hyperref}
\hypersetup{
    colorlinks=true,
    linkcolor=blue,
    filecolor=magenta,
    urlcolor=cyan,
    pdftitle={Physics 1 Notes},
    pdfauthor={Richard Whitehill},
    pdfpagemode=FullScreen
}
\urlstyle{same}

\newcommand{\eref}[1]{Eq.~(\ref{eq:#1})}
\newcommand{\erefs}[2]{Eqs.~(\ref{eq:#1})--(\ref{eq:#2})}

\newcommand{\fref}[1]{Fig.~(\ref{fig:#1})}
\newcommand{\frefs}[2]{Fig.~(\ref{fig:#1})--(\ref{fig:#2})}

\newcommand{\aref}[1]{Appendix~(\ref{app:#1})}
\newcommand{\sref}[1]{Section~(\ref{sec:#1})}
\newcommand{\srefs}[2]{Sections~(\ref{sec:#1})-(\ref{sec:#2})}

\newcommand{\tref}[1]{Table~(\ref{tab:#1})}
\newcommand{\trefs}[2]{Table~(\ref{tab:#1})--(\ref{tab:#2})}

% tcolorbox formatting/definitions
\usepackage[most]{tcolorbox}
\usepackage{xcolor}
\usepackage{xifthen}
\usepackage{parskip}

\definecolor{peach}{rgb}{1.0,0.8,0.64}

\DeclareTColorBox[auto counter, number within=chapter]{defbox}{O{}}{
    enhanced,
    boxrule=0pt,
    frame hidden,
    borderline west={4pt}{0pt}{green!50!black},
    colback=green!5,
    before upper=\textbf{Definition \thetcbcounter \ifthenelse{\isempty{#1}}{}{: #1} \\ },
    sharp corners
}

\newcommand*{\eqbox}{\tcboxmath[
    enhanced,
    colback=black!10!white,
    colframe=black,
    sharp corners,
    size=fbox,
    boxsep=8pt,
    boxrule=1pt
]}

\newtcolorbox[auto counter, number within=chapter]{exbox}{
    parbox=false,
    breakable,
    enhanced,
    sharp corners,
    boxrule=1pt,
    colback=white,
    colframe=black,
    before upper= \textbf{Example \thetcbcounter:}\,,
    before lower= \textbf{Solution:}\,,
    segmentation hidden
}

\newtcolorbox{resbox}{
    enhanced,
    colback=black!10!white,
    colframe=black,
    boxrule=1pt,
    boxsep=0pt,
    top=2pt,
    ams nodisplayskip,
    sharp corners
}


\begin{document}

\prob{1}{

Consider the Pauli matrix:
\begin{eqnarray}
    \sigma_{y} = \begin{pmatrix}
        0 & -i \\
        i & 0
    \end{pmatrix}
.\end{eqnarray}

(a) Obtain the eigenvalues and eigenvectors of $\sigma_{y}$.

(b) Find a unitary operator $U$ which diagonalizes $\sigma_{y}$.

}

\sol{

(a) We can easily see that the characteristic equation of $\sigma_{y}$ is
\begin{eqnarray}
    \lambda^2 - 1 = 0 \Rightarrow \lambda = \pm 1
.\end{eqnarray}
Thus, the components of the eigenvectors satisfy
\begin{eqnarray}
    -\lambda a_1 = i a_2 \Rightarrow a_2 = i \lambda a_1
,\end{eqnarray}
and the eigenvectors are then (up to some constant)
\begin{eqnarray}
\eqbox{
\begin{aligned}
    \begin{pmatrix}
        1 \\ i
    \end{pmatrix}
    &\leftrightarrow 1
    \\
    \begin{pmatrix}
        1 \\ -i
    \end{pmatrix}
    &\leftrightarrow -1
\end{aligned}
}
.\end{eqnarray}

(b) We can easily diagonalize $\sigma_{y}$ by writing
\begin{eqnarray}
\eqbox{
    U = \frac{1}{\sqrt{2}} \begin{pmatrix}
        1 & 1 \\
        i & -i
    \end{pmatrix}
}
,\end{eqnarray}
where
\begin{eqnarray}
    A' = U^{\dagger} A U = \begin{pmatrix}
        1 & 0 \\
        0 & -1
    \end{pmatrix}
.\end{eqnarray}

}


\prob{2}{

Consider a rotation matrix $a$ with complex elements $a_{ij}$:
\begin{eqnarray}
    a = \begin{pmatrix}
        a_{11} & a_{12} & a_{13} \\
        a_{21} & a_{22} & a_{23} \\
        a_{31} & a_{32} & a_{33}
    \end{pmatrix}
.\end{eqnarray}
Find the inverse matrix $a^{-1}$.

}

\sol{

Since $a$ is a rotation matrix, it must be unitary.
We prove this as follows.
Let us posit a defining property of rotation matrices: they preserve the norm of a vector under a transformation.
That is, $||v|| = ||R v ||$.
Using this we see that
\begin{eqnarray}
    v^{\dagger} v = v^{\dagger} ( R^{\dagger} R ) v \Rightarrow R^{\dagger} R = \mathbb{I}
.\end{eqnarray}

Another way to see this is by recalling from Quantum Mechanics that the angular momentum operator $J$ is the generator of rotations.
That is, we can define a generic rotation operator through an angle $\theta$ about an axis defined by a unit vector $\vu*{n}$ as $R = \exp(-i \theta \vu*{n} \cdot \va*{J})$, which has inverse $R^{-1} = R^{\dagger}$.
This is a more abstract object than a matrix in general: it performs a rotation of states $\ket{\psi} \rightarrow \ket{\psi'} = R \ket{\psi}$ in the Hilbert space.
We can, however, typically represent the operator $R$ using a matrix, and if our Hilbert space has dimension 3 (e.g. a spin-1 system), then our rotation matrix $R \in \mathbb{C}^{3 \times 3}$.

Finally, we write down explicitly
\begin{eqnarray}
\eqbox{
    a^{-1} = a^{\dagger} = \begin{pmatrix}
        a_{11}^{*} & a_{21}^{*} & a_{31}^{*} \\
        a_{12}^{*} & a_{22}^{*} & a_{32}^{*} \\
        a_{13}^{*} & a_{23}^{*} & a_{33}^{*}
    \end{pmatrix}
}
.\end{eqnarray}

}


\prob{3}{

Using the definition of a function of a matrix given in Lect. 22 and the properties of the Pauli matrices $\sigma_{i}^2 = \mathbb{I}$, prove the following relation:
\begin{eqnarray}
    \exp(i \theta \sigma_{z}) = \mathbb{I} \cos{\theta} + i \sigma_{z} \sin{\theta}
.\end{eqnarray}

}

\sol{

From the definition of the exponential of an operator, we can write
\begin{eqnarray}
    \exp(i \theta \sigma_{z}) = \sum_{n=0}^{\infty} \frac{1}{n!} (i \theta \sigma_{z})^{n}
.\end{eqnarray}
Let us split the sum over even and odd $n$ such that
\begin{eqnarray}
\eqbox{
\begin{aligned}
    \exp(i \theta \sigma_{z}) &= \sum_{n=0}^{\infty} \frac{1}{(2n)!} (i \theta \sigma_{z})^{2n} + \sum_{n=0}^{\infty} \frac{1}{(2n+1)!} (i \theta \sigma_{z})^{2n+1} \\
                              &= \sum_{n=0}^{\infty} \frac{(-1)^{n}}{(2n)!} \theta^{2n} \mathbb{I} + i \sum_{n=0}^{\infty} \frac{(-1)^{n}}{(2n+1)!} \theta^{2n+1} \sigma_{z} \\
                              &= \cos{\theta} \mathbb{I} + i \sin{\theta} \sigma_{z}
\end{aligned}
}
.\end{eqnarray}


}



\end{document}
